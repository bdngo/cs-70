\documentclass{article}
\usepackage{amsmath, amssymb, amsfonts, amsthm}
\usepackage{cancel}
\usepackage[output-complex-root=j]{siunitx}
\usepackage[american, nooldvoltagedirection]{circuitikz}
\usepackage{bm}
\usepackage{listings}
\usepackage{graphicx}
\usepackage{fullpage}
\usepackage{hyperref}

\renewcommand{\thesection}{\arabic{section}}
\renewcommand{\thesubsection}{\thesection.\alph{subsection}}
\renewcommand{\thesubsubsection}{\thesubsection.\roman{subsubsection}}
\renewcommand{\labelenumi}{\alph{enumi}.}

\newtheorem{genthm}{Theorem}

\newcommand{\N}{\mathbb{N}}
\newcommand{\Z}{\mathbb{Z}}
\newcommand{\Q}{\mathbb{Q}}
\newcommand{\R}{\mathbb{R}}
\renewcommand{\C}{\mathbb{C}}
\newcommand{\unit}[1]{\bm{\hat{#1}}}
\newcommand{\iprod}[2]{\left\langle #1, #2 \right\rangle}
\newcommand{\tpose}[1]{\left[#1\right]^{\! \top} \!\!}
\newcommand{\diff}[1]{\frac{d}{d #1}}

\lstset{
    language=Python,
    tabsize=4,
    basicstyle=\ttfamily,
    numbers=left,
    numberstyle=\ttfamily,
    keywordstyle=\color{blue},
    frame=single
}

\title{CS70 HW01}
\author{Bryan Ngo}
\date{2020-08-30}

\begin{document}

\maketitle

I did this homework by myself.

\section{Calculus Review}

\subsection{}

\begin{equation}
    \sum_{k = 1}^\infty \frac{9}{2^k} = 9 \sum_{k = 1}^\infty \frac{1}{2^k} = 9
\end{equation}

\subsection{}

\begin{equation}
    S = \sum_{i = 0}^n (2n + 1)
\end{equation}

\subsection{}

\begin{equation}
    I = \int_0^{\infty} \sin(t) e^{-t} \, dt
\end{equation}
Let \(u = \sin(t)\) and \(dv = e^{-t}\), then \(du = \cos(t)\) and \(v = -e^{-t}\),
\begin{equation}
    I = -\sin(t) e^{-t}|_0^\infty + \int_0^\infty \cos(t) e^{-t} \, dt
\end{equation}
Once more, let \(u = \cos(t)\) and \(dv = e^{-t}\).
Therefore, \(du = -\sin(t)\) and \(v = -e^{-t}\),
\begin{align}
    I &= \lim_{R \to \infty} (-\sin(R) e^{-R} + \sin(0) e^{0}) + \int_0^\infty \cos(t) e^{-t} \, dt \\
    &= -\cos(t) e^{-t}|_0^\infty - \underbrace{\int_0^\infty \sin(t) e^{-t} \, dt}_I \\
    \Rightarrow 2I &= \lim_{R \to \infty} (-\cos(R) e^{-R} + \cos(0) e^0) \\
    &= 1 \Rightarrow I = \frac{1}{2}
\end{align}

\subsection{}

\begin{equation}
    f(x) = -x \ln(x)
\end{equation}
To find the max, we first find the derivative with respect to \(x\),
\begin{equation}
    \diff{x} (-x \ln(x)) = (-\ln(x)) + (-1) = -\ln(x) - 1
\end{equation}
Setting \(f'(x) = 0\),
\begin{equation}
    -\ln(x) - 1 = 0 \implies x = \frac{1}{e}
\end{equation}
Since \(x = 0\) is not in the domain of \(f(x)\), we do not need to consider it.
Using the second derivative test,
\begin{equation}
    f''(x)|_{x = \frac{1}{e}} = \left.-\frac{1}{x}\right|_{x = \frac{1}{e}} = -e
\end{equation}
Since \(f''\left(\frac{1}{e}\right) < 0\), the value is a maximum.
Thus, \(f(x)\) has a maximum at \(x = \frac{1}{e}\) with a value of \(f\left(\frac{1}{e}\right) = \frac{1}{e}\).

\section{Propositional Practice}

\begin{enumerate}
    \item \(|\{x \in \R | x^2 = 0\}| = 1\)
    \item \((\forall a, b \in \Q)(\exists c \in \Q)(a < b \implies a < c < b)\)
    \item \((\forall x \in \Z)(x^2 > 4 \implies x > 2 \lor x < -2)\)
    \item All real numbers are also complex numbers.
    \item The difference of the squares of any two integers will never be equal to \(10\).
    \item Any even natural number can be represented as the sum of two primes.
\end{enumerate}

\end{document}
