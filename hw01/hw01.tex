\documentclass{article}
\usepackage{amsmath, amssymb, amsfonts, amsthm}
\usepackage{cancel}
\usepackage[output-complex-root=j]{siunitx}
\usepackage[american, nooldvoltagedirection]{circuitikz}
\usepackage{bm}
\usepackage{listings}
\usepackage{graphicx}
\usepackage{fullpage}
\usepackage{hyperref}

\renewcommand{\thesection}{\arabic{section}}
\renewcommand{\thesubsection}{\thesection.\alph{subsection}}
\renewcommand{\thesubsubsection}{\thesubsection.\roman{subsubsection}}
\renewcommand{\labelenumi}{\alph{enumi}.}
\newcommand{\lemmaautorefname}{Lemma}

\newtheorem{theorem}{Theorem}
\newtheorem{lemma}{Lemma}

\newcommand{\N}{\mathbb{N}}
\newcommand{\Z}{\mathbb{Z}}
\newcommand{\Q}{\mathbb{Q}}
\newcommand{\R}{\mathbb{R}}
\renewcommand{\C}{\mathbb{C}}
\newcommand{\unit}[1]{\bm{\hat{#1}}}
\newcommand{\iprod}[2]{\left\langle #1, #2 \right\rangle}
\newcommand{\tpose}[1]{\left[#1\right]^{\! \top} \!\!}
\newcommand{\diff}[1]{\frac{d}{d #1}}

\lstset{
    language=Python,
    tabsize=4,
    basicstyle=\ttfamily,
    numbers=left,
    numberstyle=\ttfamily,
    keywordstyle=\color{blue},
    frame=single
}

\title{CS70 HW01}
\author{Bryan Ngo}
\date{2020-08-30}

\begin{document}

\maketitle

\section{Calculus Review}

\subsection{}

\begin{equation}
    \sum_{k = 1}^\infty \frac{9}{2^k} = 9 \sum_{k = 1}^\infty \frac{1}{2^k} = 9 \frac{\frac{1}{2}}{1 - \frac{1}{2}} = 9
\end{equation}

\subsection{}

\begin{equation}
    S = \sum_{i = 0}^n (2i + 1)
\end{equation}

\subsection{}

\begin{equation}
    I = \int_0^{\infty} \sin(t) e^{-t} \, dt
\end{equation}
Let \(u = \sin(t)\) and \(dv = e^{-t}\), then \(du = \cos(t)\) and \(v = -e^{-t}\),
\begin{equation}
    I = -\sin(t) e^{-t}|_0^\infty + \int_0^\infty \cos(t) e^{-t} \, dt
\end{equation}
Once more, let \(u = \cos(t)\) and \(dv = e^{-t}\).
Therefore, \(du = -\sin(t)\) and \(v = -e^{-t}\),
\begin{align}
    I &= \lim_{R \to \infty} (-\sin(R) e^{-R} + \sin(0) e^{0}) + \int_0^\infty \cos(t) e^{-t} \, dt \\
    &= -\cos(t) e^{-t}|_0^\infty - \underbrace{\int_0^\infty \sin(t) e^{-t} \, dt}_I \\
    \Rightarrow 2I &= \lim_{R \to \infty} (-\cos(R) e^{-R} + \cos(0) e^0) \\
    &= 1 \implies I = \frac{1}{2}
\end{align}

\subsection{}

\begin{equation}
    f(x) = -x \ln(x)
\end{equation}
To find the max, we first find the derivative with respect to \(x\),
\begin{equation}
    \diff{x} (-x \ln(x)) = (-\ln(x)) + (-1) = -\ln(x) - 1
\end{equation}
Setting \(f'(x) = 0\),
\begin{equation}
    -\ln(x) - 1 = 0 \implies x = \frac{1}{e}
\end{equation}
Since \(x = 0\) is not in the domain of \(f(x)\), we do not need to consider it.
Using the second derivative test,
\begin{equation}
    f''(x)|_{x = \frac{1}{e}} = \left.-\frac{1}{x}\right|_{x = \frac{1}{e}} = -e
\end{equation}
Since \(f''\left(\frac{1}{e}\right) < 0\), the value is a maximum.
Thus, \(f(x)\) has a maximum at \(x = \frac{1}{e}\) with a value of \(f\left(\frac{1}{e}\right) = \frac{1}{e}\).

\section{Propositional Practice}

\begin{enumerate}
    \item \(|\{x \in \R | x^2 = 0\}| = 1\)
    \item \((\forall a, b \in \Q)(\exists c \in \Q)(a < b \implies a < c < b)\)
    \item \((\forall x \in \Z)(x^2 > 4 \implies (x > 2) \lor (x < -2))\)
    \item All real numbers are also complex numbers.
    \item The difference of the squares of any two integers will never be equal to \(10\).
    \item Any even natural number can be represented as the sum of two primes.
\end{enumerate}

\section{Tautologies and Contradictions}

\subsection{}

\begin{equation}
    \begin{array}{c|c|c}
        P & Q & P \implies (Q \land P) \lor (\lnot Q \land P) \\
        \hline
        T & T & T \\
        T & F & T \\
        F & T & T \\
        F & F & T
    \end{array}
\end{equation}
Tautology

\subsection{}

\begin{equation}
    \begin{array}{c|c|c}
        P & Q & (P \lor Q) \lor (P \lor \lnot Q) \\
        \hline
        T & T & T \\
        T & F & T \\
        F & T & T \\
        F & F & T
    \end{array}
\end{equation}
Tautology

\subsection{}

\begin{equation}
    \begin{array}{c|c|c}
        P & Q & P \land (P \implies \lnot Q) \land Q \\
        \hline
        T & T & F \\
        T & F & F \\
        F & T & F \\
        F & F & F
    \end{array}
\end{equation}
Contradiction

\subsection{}

\begin{equation}
    \begin{array}{c|c|c}
        P & Q & (\lnot P \implies Q) \implies (\lnot Q \implies P) \\
        \hline
        T & T & T \\
        T & F & T \\
        F & T & T \\
        F & F & T
    \end{array}
\end{equation}
Tautology

\subsection{}

\begin{equation}
    \begin{array}{c|c|c}
        P & Q & (\lnot P \implies \lnot Q) \land (P \implies \lnot Q) \land Q \\
        \hline
        T & T & F \\
        T & F & F \\
        F & T & F \\
        F & F & F
    \end{array}
\end{equation}
Contradiction

\subsection{}

\begin{equation}
    \begin{array}{c|c|c}
        P & Q & (\lnot (P \land Q)) \land (P \lor Q) \\
        \hline
        T & T & F \\
        T & F & T \\
        F & T & T \\
        F & F & F
    \end{array}
\end{equation}
Neither

\section{Prove or Disprove}

\subsection{}

\begin{theorem}
    \((\forall n \in \N)\) if \(n\) is odd then \(n^2 + 4n\) is odd.
\end{theorem}
\begin{proof}
    By definition, any odd natural number can be written in the form \(2k + 1\), where \(k \in \N\).
    Plugging in,
    \begin{equation}
        (2k + 1)^2 + 4(2k + 1) = 4k^2 + 4k + 1 + 8k + 4 = 4k^2 + 12k + 5 = 2(2k^2 + 6k + 1) + 1
    \end{equation}
    \(2k^2 + 6k + 1 \in \N\) for natural \(k\) due to closure under addition, multiplication, and natural exponentiation.
    Therefore, the overall result is odd.
\end{proof}

\subsection{}

\begin{theorem}
    \((\forall a, b \in \R)\) if \(a + b \leqslant 15\) then \(a \leqslant 11\) or \(b \leqslant 4\).
\end{theorem}
\begin{proof}
    By contraposition,
    \begin{align}
        (\forall a, b \in \R)((a + b \leqslant 15) &\implies (a \leqslant 11) \lor (b \leqslant 4)) \\
        \Rightarrow (\forall a, b \in \R)(\lnot ((a \leqslant 11) \lor (b \leqslant 4)) &\implies (a + b > 15)) \\
        \Rightarrow (\forall a, b \in \R)((a > 11) \land (b > 4) &\implies (a + b > 15))
    \end{align}
    The proof holds because since \(a > 11\) and \(b > 4\), their sum must be larger than their upper limit, that being 15.
\end{proof}

\subsection{}

\begin{theorem}
    \(\forall r \in \R\) if \(r^2\) is irrational, then \(r\) is irrational.
\end{theorem}
\begin{proof}
    By contraposition, if \(r\) is rational, then \(r^2\) is rational.
    Let \(r = \frac{p}{q}\), where \(p, q \in \Z\), \(q \neq 0\).
    Then, \(r^2 = \left(\frac{p}{q}\right)^2 = \frac{p^2}{q^2}\).
    Since the rationals are closed under exponentiation by integers, \(r^2 \in \Q\).
\end{proof}

\subsection{}

\begin{theorem}
    \(\forall n \in \Z^+\) \(5n^3 > n!\).
\end{theorem}
\begin{proof}
    \textbf{Counterexample}: Let \(n = 7\).
    \(5 (7)^3 = 1715 < 7! = 5040.\)
\end{proof}

\section{Twin Primes}

\subsection{}

\begin{theorem}
    Let \(p > 3\) be some prime.
    Then, \(p = 3k \pm 1\) for some \(k \in \Z\).
\end{theorem}
\begin{proof}
    By contraposition, if \(p\) cannot be represented in the form \(3k \pm 1\), it then \(p\) is not prime.
    We can prove that all natural numbers greater than \(3\) assume the form \(\{3k \pm 1, 3k\}\):
    \begin{lemma} \label{twin-prime-lemma}
        \((\forall n \in \N) (\exists k \in \N) ((n > 3) \implies (n = 3k \pm 1) \lor (3 | n))\)
    \end{lemma}
    \begin{proof}
        \begin{equation}
            \begin{array}{c|c}
                n & \text{Form} \\
                \hline
                4 & 3k + 1 \\
                5 & 3k - 1 \\
                6 & 3k \\
                7 & 3k + 1 \\
                8 & 3k - 1 \\
                9 & 3k \\
                \vdots & \vdots
            \end{array}
        \end{equation}
        As observed, a pattern emerges.
        Let \(q\) be some natural number such that \(3 | q\).
        We can get the numbers one greater and one less than \(q\) by doing \(3k \pm 1\).
        We now have a collection of 3 naturals, so the cycle repeats.
    \end{proof}
    By \autoref{twin-prime-lemma}, all integers can be represented in the three forms.
    However, the only way for \(p\) not be represented in the form \(3k \pm 1\) is for \(p = 3k\).
    This means that \(3 | p\) and is not prime.
\end{proof}

\subsection{}

\begin{theorem}
    5 is the only prime number that is in two different twin prime pairs.
\end{theorem}
\begin{proof}
    Let \(a, b, p\) be prime numbers.
    Then, the relationship as stated in the theorem is
    \begin{equation}
        a - p = p - b = 2
    \end{equation}
    We can prove by cases that \((p \nless 5) \land (p \ngtr 5) \implies (p = 5)\).
    For \(p < 5\), there is only \(\{2, 3\}\).
    \(2\) is not a twin prime, and \(3\) is only a twin prime with \(5\). \\
    For \(p > 5\), if the difference between the primes is two, then if \(a\) or \(b\) is \(3k \pm 1\), then \(p = 3k \mp 1\).
    It is important to note that \(a\) and \(b\) cannot both be \(3k \pm 1\) because they are by definition, distinct, leading to two subcases:
    \begin{align}
        (3k + 1) - (3k - 1) = (3k - 1) - (3k - 1) &= 2 \\
        2 \overset{?}{=} 0 &\overset{?}{=} 2 \\
        (3k - 1) - (3k + 1) = (3k + 1) - (3k + 1) &= 2 \\
        -2 \overset{?}{=} 0 &\overset{?}{=} 2
    \end{align}
    This means it is impossible for any \(p > 5\) to be in two twin primes so long as \(p > 3\).
\end{proof}

\section{Social Network}

\subsection{}

\begin{theorem}
    For all cases with \(n = 5\) people, there exists a group of 3 full friends or full strangers.
\end{theorem}
\begin{proof}
    \textbf{Counterexample}: Consider the following adjacency list: \\
    \begin{tabular}{c|c|c}
        Person & Friends & Strangers \\
        \hline
        1 & 2, 5 & 3, 4 \\
        2 & 1, 3 & 4, 5 \\
        3 & 2, 4 & 1, 5 \\
        4 & 3, 5 & 1, 2 \\
        5 & 1, 4 & 2, 3
    \end{tabular} \\
    When drawn graphically, there are no triangles surrounded by \(F\) or \(S\) edges.
    By inspection, there are no 3-cycles of either friends or strangers.
    The equation governing the friends and strangers of a given person can be described as
    \begin{equation}
        f + s = n - 1
    \end{equation}
    where \(f, s, n\) is the number of friends, strangers, and total people, respectively.
    Therefore, for \(n = 5\),
    \begin{equation}
        f + s = 4
    \end{equation}
    This means that it is possible for someone to have less than three friends/strangers each.
\end{proof}

\subsection{}

\begin{theorem}
    For all cases with \(n = 6\) people, there exists a group of 3 full friends or full strangers.
\end{theorem}
\begin{proof}
    The equation governing the friends and strangers of a given person can be described as
    \begin{equation}
        f + s = n - 1
    \end{equation}
    where \(f, s, n\) is the number of friends, strangers, and total people, respectively.
    Therefore, for \(n = 6\),
    \begin{equation}
        f + s = 5
    \end{equation}
    The number of friends/strangers cannot go below 3 without the other going above 3.
    This means that only one friendship or strangership between those 3+ friends/strangers is required to satisfy the theorem.
    In the case that none of them share that, they themselves form their own 3-cycle.
    Therefore, by the pigeonhole principle, a person will always have at least 3-cycle.
\end{proof}

\section{Preserving Set Operations}

\subsection{}

\begin{theorem}
    \(f^{-1}(A \cup B) = f^{-1}(A) \cup f^{-1}(B)\).
\end{theorem}
\begin{proof}
    Suppose \(x \in f^{-1}(A \cup B)\).
    Then,
    \begin{align}
        f(x) &\in A \cup B \\
        \implies (f(x) \in A) &\lor (f(x) \in B) \\
        (x \in f^{-1}(A)) &\lor (x \in f^{-1}(B)) \\
        \implies x \in f^{-1}(A) &\cup f^{-1}(B) \\
        \implies f^{-1}(A \cup B) &\subseteq f^{-1}(A) \cup f^{-1}(B)
    \end{align}
    Now, let \(x \in f^{-1}(A) \cup f^{-1}(B)\).
    Then,
    \begin{align}
        x &\in f^{-1}(A) \\
        \implies x &\in f^{-1}(A \cup B) 
    \end{align}
    The argument is similar for set \(B\).
    Therefore, \(f^{-1}(A) \cup f^{-1}(B) \subseteq f^{-1}(A \cup B)\), and the sets are equal.
\end{proof}

\subsection{}

\begin{theorem}
    \(f^{-1}(A \cap B) = f^{-1}(A) \cap f^{-1}(B)\).
\end{theorem}
\begin{proof}
    Suppose \(x \in f^{-1}(A \cap B)\).
    Then,
    \begin{align}
        f(x) &\in A \cap B \\
        \implies (f(x) \in A) &\land (f(x) \in B) \\
        (x \in f^{-1}(A)) &\land (x \in f^{-1}(B)) \\
        \implies x \in f^{-1}(A) &\cap f^{-1}(B) \\
        \implies f^{-1}(A \cap B) &\subseteq f^{-1}(A) \cap f^{-1}(B)
    \end{align}
    Now, let \(x \in f^{-1}(A) \cap f^{-1}(B)\).
    Then,
    \begin{align}
        (x \in f^{-1}(A)) &\land (x \in f^{-1}(B)) \\
        (f(x) \in A) &\land (f(x) \in B) \\
        \implies f(x) &\in A \cap B \\
        x &\in f^{-1}(A \cap B)
    \end{align}
    Therefore, \(f^{-1}(A) \cap f^{-1}(B) \subseteq f^{-1}(A \cap B)\), and the sets are equal.
\end{proof}

\subsection{}

\begin{theorem}
    \(f^{-1}(A \setminus B) = f^{-1}(A) \setminus f^{-1}(B)\).
\end{theorem}
\begin{proof}
    Suppose \(x \in f^{-1}(A \setminus B)\).
    Then,
    \begin{align}
        f(x) &\in A \setminus B \\
        \implies (f(x) \in A) &\land (f(x) \notin B) \\
        (x \in f^{-1}(A)) &\land (x \notin f^{-1}(B)) \\
        \implies x \in f^{-1}(A) &\setminus f^{-1}(B) \\
        \implies f^{-1}(A \setminus B) &\subseteq f^{-1}(A) \setminus f^{-1}(B)
    \end{align}
    Now, let \(x \in f^{-1}(A) \setminus f^{-1}(B)\).
    Then,
    \begin{align}
        (x \in f^{-1}(A)) &\land (x \notin f^{-1}(B)) \\
        (f(x) \in A) &\land (f(x) \notin B) \\
        \implies f(x) &\in A \setminus B \\
        x &\in f^{-1}(A \setminus B)
    \end{align}
    Therefore, \(f^{-1}(A) \setminus f^{-1}(B) \subseteq f^{-1}(A \setminus B)\), and the sets are equal.
\end{proof}

\subsection{}

\begin{theorem}
    \(f(A \cup B) = f(A) \cup f(B)\).
\end{theorem}
\begin{proof}
    Suppose \(x \in A \cup B\).
    Then,
    \begin{align}
        (f(x) \in f(A)) &\lor (f(x) \in f(B)) \\
        \implies f(x) &\in f(A) \cup f(B) \\
        \implies f(A \cup B) &\subseteq f(A) \cup f(B) \\
    \end{align}
    Now, let \(y \in f(A) \cup f(B)\) with \(f(x) = y\).
    Then,
    \begin{align}
        (y \in f(A)) &\lor (y \in f(B)) \\
        \implies (x \in A) &\lor (x \in B) \\
        \implies x &\in A \cup B \\
        \implies f(x) = y &\in f(A \cup B)
    \end{align}
    Then, \(f(A) \cup f(B) \subseteq f(A \cup B)\), and the sets are equal.
\end{proof}

\subsection{}

\begin{theorem}
    \(f(A \cap B) \subseteq f(A) \cap f(B)\).
\end{theorem}
\begin{proof}
    Suppose \(x \in A \cap B\).
    Then,
    \begin{align}
        (f(x) \in f(A)) &\land (f(x) \in f(B)) \\
        \implies f(x) &\in f(A) \cap f(B) \\
        \implies f(A \cap B) &\subseteq f(A) \cap f(B) \\
    \end{align}
\end{proof}
\textbf{Counterexample}: 

\section{Sundry}

I did homework with:
\begin{itemize}
    \item Manny Fuentes \href{mailto:mfuentes@berkeley.edu}{mfuentes@berkeley.edu}
    \item Avery Perez \href{mailto:aj_perez@berkeley.edu}{aj\_perez@berkeley.edu}
    \item Victor Zhang \href{mailto:vzhang6@berkeley.edu}{vzhang6@berkeley.edu}
\end{itemize}

\end{document}
