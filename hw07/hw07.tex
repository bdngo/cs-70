\documentclass{article}
\usepackage{amsmath, amssymb, amsfonts, amsthm}
\usepackage{cancel}
\usepackage[output-complex-root=j]{siunitx}
\usepackage[american, nooldvoltagedirection]{circuitikz}
\usepackage{bm}
\usepackage{listings}
\usepackage{graphicx}
\usepackage{fullpage}
\usepackage{hyperref}

\renewcommand{\thesection}{\arabic{section}}
\renewcommand{\thesubsection}{\thesection.\alph{subsection}}
\renewcommand{\thesubsubsection}{\thesubsection.\roman{subsubsection}}
\newcommand{\lemmaautorefname}{Lemma}

\newtheorem{theorem}{Theorem}
\newtheorem{lemma}{Lemma}

\newcommand{\N}{\mathbb{N}}
\newcommand{\Z}{\mathbb{Z}}
\newcommand{\Q}{\mathbb{Q}}
\newcommand{\R}{\mathbb{R}}
\renewcommand{\C}{\mathbb{C}}
\newcommand{\unit}[1]{\bm{\hat{#1}}}
\newcommand{\iprod}[2]{\left\langle #1, #2 \right\rangle}
\newcommand{\tpose}[1]{\left[#1\right]^{\! \top} \!\!}
\newcommand{\diff}[1]{\frac{d}{d #1}}

\lstset{
    language=Python,
    tabsize=4,
    basicstyle=\ttfamily,
    numbers=left,
    numberstyle=\ttfamily,
    keywordstyle=\color{blue},
    frame=single
}

\title{CS 70 HW 07}
\author{Bryan Ngo}
\date{2020-10-15}

\begin{document}

\maketitle

\section{Strings}

Given a string of length \(5\), the number of strings purely containing \(\{A, B, C\}\) is \(3^5 = 243\), since each letter can be either \(\{A, B, C\}\).
As for the second part, we take a subset of the first part, except we now subtract all possibilities where no more than \(2\) of each character show up, multiplied by each character.
Then, we have to add back in the cases where they are all the same character, bringing us to \(3^5 - 3 \cdot 2^5 + 3 = 150\).

\section{Palindromes}

Since the middle digit is free, we have \(10\) possibilities for the middle digit.
Then, we have the other two to choose.
Since by definition they have to be the same on either side, there will be two of each digit.
However, the first digit cannot be zero so there are only \(9\) possibilities for the first and last digit.
Therefore, the final count is \(10 \cdot 10 \cdot 9 = 900\).

\section{Maze in General and Trees Too!!!}

\subsection{}

Since there are \(2n\) possible moves to make, and each axis requires \(n\) moves, the total number is \(\binom{2n}{n}\).

\subsection{}

The total number of possible moves is still \(n - 1 + n + 1 = 2n\).
However, there is an uneven axis, which means that each axis can either be \(n \pm 1\) moves, bringing us to \(\binom{2n}{n \pm 1}\).

\subsection{}

\subsubsection{}

The total number of paths is
\begin{equation}
    \binom{2n}{n} - \binom{2n}{n + 1}
\end{equation}
This is because we exclude all the paths where you go from \((i, i) \to (i, i + 1)\), which would imply crossing the \(y = x\) line.

\subsubsection{}

Consider two subproblems: the number of paths going from \((0, 0) \to (i, i)\) and from \((i, i) \to (n, n)\).
By definition, the number of paths in the first set is \(F_i\).
Then, the remaining number of paths is \(F_{n - i - 1}\), since that is the "distance" between the locations.
Then, the total number of paths is simply the product of the two, or \(F_i \cdot F_{n - i - 1}\).

\subsubsection{}

% \begin{theorem}
%     \begin{equation}
%         T_n = 2 T_{n - 2} + 1
%     \end{equation}
%     where \(n \geqslant 3\).
% \end{theorem}
% \begin{proof}
%     Base case: \(n = 3\), \(T_3 = 2 (0) + 1\), there is only one spanning binary tree since \(T_2\) does not yield a valid tree.
%     Inductive hypothesis: \(T_k = 2 T_{k - 2} + 1\).
%     We must add one for the case in which the tree is balanced.
%     Then, we take the previous tree, and mirror the new node across either the left or right subtree.
%     This means that there are two new trees for each one in the previous step, which completes the formula.
% \end{proof}
Consider \(n\) nodes.
Eliminate one of the nodes since it will be the root.
Consider two sets, the first \(i\) and the \(n - i - 1\).
Then, the \(i\) paths represent the left subtrees.
Then, the \(n - i - 1\) paths represent the right subtrees.
Then, the total number of binary spanning trees is
\begin{equation}
    T_n = \sum_{i = 1}^{n - 1} T_i \cdot T_{n - i - 1}
\end{equation}
We can think of the binary spanning tree with the left path represents an \(x\)-axis move and the right path represents \(y\)-axis move.
The root node represents the starting point.
This makes sense as given the conditions of never crossing \(y = x\), at each point there are only two paths, either increase \(x\) or increase \(y\).

\section{Sundry}

I did homework with:
\begin{itemize}
    \item Manny Fuentes \href{mailto:mfuentes@berkeley.edu}{mfuentes@berkeley.edu}
    \item Avery Perez \href{mailto:aj_perez@berkeley.edu}{aj\_perez@berkeley.edu}
    \item Victor Zhang \href{mailto:vzhang6@berkeley.edu}{vzhang6@berkeley.edu}
    \item Alex Chuang \href{mailto:alexc0413@berkeley.edu}{alexc0413@berkeley.edu}
\end{itemize}

\end{document}
