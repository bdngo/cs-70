\documentclass{article}
\usepackage{amsmath, amssymb, amsfonts, amsthm}
\usepackage{cancel}
\usepackage[output-complex-root=j]{siunitx}
\usepackage[american, nooldvoltagedirection]{circuitikz}
\usepackage{bm}
\usepackage{listings}
\usepackage{graphicx}
\usepackage{fullpage}
\usepackage{hyperref}

\renewcommand{\thesection}{\arabic{section}}
\renewcommand{\thesubsection}{\thesection.\alph{subsection}}
\renewcommand{\thesubsubsection}{\thesubsection.\roman{subsubsection}}

\newtheorem{theorem}{Theorem}

\newcommand{\N}{\mathbb{N}}
\newcommand{\Z}{\mathbb{Z}}
\newcommand{\Q}{\mathbb{Q}}
\newcommand{\R}{\mathbb{R}}
\renewcommand{\C}{\mathbb{C}}
\newcommand{\unit}[1]{\bm{\hat{#1}}}
\newcommand{\iprod}[2]{\left\langle #1, #2 \right\rangle}
\newcommand{\tpose}[1]{\left[#1\right]^{\! \top} \!\!}
\newcommand{\diff}[1]{\frac{d}{d #1}}

\lstset{
    language=Python,
    tabsize=4,
    basicstyle=\ttfamily,
    numbers=left,
    numberstyle=\ttfamily,
    keywordstyle=\color{blue},
    frame=single
}

\title{CS 70 HW04}
\author{Bryan Ngo}
\date{2020-09-22}

\begin{document}

\maketitle

\section{Fibonacci GCD}

\begin{theorem}
    For all \(n \geqslant 0\), \(\gcd(F_n, F_{n - 1}) = 1\), where \(F_n\) is the \(n\)-th Fibonacci number.
\end{theorem}
\begin{proof}
    Base case: \(\gcd(F_1, F_0) = 1\).
    Inductive hypothesis: For all \(k \geqslant 0\), \(\gcd(F_k, F_{k - 1}) = 1\).
    \begin{equation}
        \gcd(F_{k + 1}, F_k) = \gcd(F_k + F_{k - 1}, F_k) = \gcd(F_k, (F_k + F_{k - 1}) \bmod F_k) = \underbrace{\gcd(F_k, F_{k - 1})}_{\text{inductive hypothesis}} = 1
    \end{equation}
\end{proof}

\section{The Last Digit}

\subsection{}

\begin{align}
    9k + 5 &\equiv 2k + 1 \pmod{10} \\
    \Rightarrow 7k &\equiv -4 \equiv 6 \pmod{10}
\end{align}
By inspection, we find that the \(3 \equiv 7^{-1} \pmod{10}\).
Then, \(k = 3 \cdot 6 = 18 \equiv 8 \pmod{10}\).

\subsection{}

\begin{equation}
    S = \sum_{i = 1}^{19} i!
\end{equation}
For all \(n > 4\), \(n! \equiv 0 \pmod{10}\).
Therefore, we can ignore it.
With this in mind, \(S^2 \equiv (1! + 2! + 3! + 4!)^2 \equiv 33^2 \equiv 9 \pmod{10}\).

\section{Celebrate and Remember Textiles}

We set up the Chinese remainder theorem system
\begin{align}
    x &\equiv 2 \pmod{4} \\
    x &\equiv 2 \pmod{5} \\
    x &\equiv 4 \pmod{7}
\end{align}
For modulo \(4\),
\begin{equation}
    35k \equiv 3k \equiv 2 \pmod{4}
\end{equation}
By inspection, we find that \(k = 2\).
For modulo \(5\),
\begin{equation}
    28k \equiv 3k \equiv 2 \pmod{5}
\end{equation}
By inspection, we find that \(k = 4\).
For modulo \(7\),
\begin{equation}
    20k \equiv 6k \equiv 4 \pmod{7}
\end{equation}
By inspection, we find that \(k = 3\).
Therefore,
\begin{equation}
    x \equiv 35 \cdot 2 + 28 \cdot 4 + 20 \cdot 3 \equiv 242 \equiv 102 \pmod{140}
\end{equation}

\section{Sparsity of Primes}

\begin{theorem}
    For any \(k \in \N\), there exists \(k\) consecutive positive integers such that none are prime powers.
\end{theorem}
\begin{proof}

\end{proof}

\section{Fermat's Little Theorem}

\begin{theorem}
    \(42 | n^7 - n, n \in \N\)
\end{theorem}
\begin{proof}
    \begin{align}
        42 | n^7 - n \implies n^7 - n &\equiv 0 \pmod{42} \\
        n (n^6 - 1) &\equiv \pmod{}
    \end{align}
\end{proof}

\end{document}
