\documentclass{article}
\usepackage{amsmath, amssymb, amsfonts, amsthm}
\usepackage{cancel}
\usepackage[output-complex-root=j]{siunitx}
\usepackage[american, nooldvoltagedirection]{circuitikz}
\usepackage{bm}
\usepackage{listings}
\usepackage{graphicx}
\usepackage{fullpage}
\usepackage{hyperref}

\renewcommand{\thesection}{\arabic{section}}
\renewcommand{\thesubsection}{\thesection.\alph{subsection}}
\renewcommand{\thesubsubsection}{\thesubsection.\roman{subsubsection}}
\newcommand{\lemmaautorefname}{Lemma}

\newtheorem{theorem}{Theorem}
\newtheorem{lemma}{Lemma}

\newcommand{\N}{\mathbb{N}}
\newcommand{\Z}{\mathbb{Z}}
\newcommand{\Q}{\mathbb{Q}}
\newcommand{\R}{\mathbb{R}}
\renewcommand{\C}{\mathbb{C}}
\newcommand{\unit}[1]{\bm{\hat{#1}}}
\newcommand{\iprod}[2]{\left\langle #1, #2 \right\rangle}
\newcommand{\tpose}[1]{\left[#1\right]^{\! \top} \!\!}
\newcommand{\diff}[1]{\frac{d}{d #1}}

\lstset{
    language=Python,
    tabsize=4,
    basicstyle=\ttfamily,
    numbers=left,
    numberstyle=\ttfamily,
    keywordstyle=\color{blue},
    frame=single
}

\title{CS 70 HW04}
\author{Bryan Ngo}
\date{2020-09-22}

\begin{document}

\maketitle

\section{Fibonacci GCD}

\begin{theorem}
    For all \(n \geqslant 0\), \(\gcd(F_n, F_{n - 1}) = 1\), where \(F_n\) is the \(n\)-th Fibonacci number.
\end{theorem}
\begin{proof}
    Base case: \(\gcd(F_1, F_0) = 1\).
    Inductive hypothesis: For all \(k \geqslant 0\), \(\gcd(F_k, F_{k - 1}) = 1\).
    \begin{equation}
        \gcd(F_{k + 1}, F_k) = \gcd(F_k + F_{k - 1}, F_k) = \gcd(F_k, (F_k + F_{k - 1}) \bmod F_k) = \underbrace{\gcd(F_k, F_{k - 1})}_{\text{inductive hypothesis}} = 1
    \end{equation}
\end{proof}

\section{The Last Digit}

\subsection{}

\begin{align}
    9k + 5 &\equiv 2k + 1 \pmod{10} \\
    \Rightarrow 7k &\equiv -4 \equiv 6 \pmod{10}
\end{align}
By inspection, we find that the \(3 \equiv 7^{-1} \pmod{10}\).
This is because \(3 \cdot 7 = 21 \equiv 0 \pmod{10}\).
Then, \(k = 3 \cdot 6 = 18 \equiv 8 \pmod{10}\).

\subsection{}

\begin{equation}
    S = \sum_{i = 1}^{19} i!
\end{equation}
For all \(n > 4\), \(n! \equiv 0 \pmod{10}\).
This fact can be trivially proven since \(5! = 120 = 12 \cdot 10\), then every number after that will be a multiple of \(10\).
Therefore,
\begin{equation}
    \sum_{i = 1}^{19} i! = 1! + 2! + 3! + 4! + \sum_{i = 4}^{19} i! \equiv 1! + 2! + 3! + 4! \pmod{10}
\end{equation}
With this in mind, \(S^2 \equiv (1! + 2! + 3! + 4!)^2 \equiv 33^2 \equiv 9 \pmod{10}\).

\section{Celebrate and Remember Textiles}

We set up the Chinese remainder theorem system
\begin{align}
    x &\equiv 2 \pmod{4} \\
    x &\equiv 2 \pmod{5} \\
    x &\equiv 4 \pmod{7}
\end{align}
For modulo \(4\),
\begin{equation}
    35k \equiv 3k \equiv 2 \pmod{4}
\end{equation}
By inspection, we find that \(k = 2\).
This is because \(3 \cdot 2 = 6 \equiv 2 \pmod{4}\).
For modulo \(5\),
\begin{equation}
    28k \equiv 3k \equiv 2 \pmod{5}
\end{equation}
By inspection, we find that \(k = 4\).
This is because \(3 \cdot 4 = 12 \equiv 2 \pmod{5}\).
For modulo \(7\),
\begin{equation}
    20k \equiv 6k \equiv 4 \pmod{7}
\end{equation}
By inspection, we find that \(k = 3\).
This is because \(6 \cdot 3 = 18 \equiv 4 \pmod{7}\).
Therefore,
\begin{equation}
    x \equiv 35 \cdot 2 + 28 \cdot 4 + 20 \cdot 3 = 242 \equiv 102 \pmod{140}
\end{equation}

\section{Sparsity of Primes}

\begin{theorem}
    For any \(k \in \N\), there exists \(k\) consecutive positive integers such that none are prime powers.
\end{theorem}
\begin{proof}
    Consider the set \(\{p_1, p_2, \ldots, p_{2k}\}\), or the set of prime numbers of length \(2k\).
    Let \(x\) be some non-prime power natural number.
    By definition, not being a prime power necessarily means that it is the product of at least two distinct primes.
    Then, the proof necessitates the modular system of equations
    \begin{align}
        x &\equiv 0 \pmod{p_1 p_2} \\
        x + 1 &\equiv 0 \pmod{p_3 p_4} &&\implies x \equiv -1 \pmod{p_3 p_4} \\
        x + 2 &\equiv 0 \pmod{p_5 p_6} &&\implies x \equiv -2 \pmod{p_5 p_6} \\
       &\vdots &&\vdots \\
        x + (k - 1) &\equiv 0 \pmod{p_{2n - 1} p_{2n}} &&\implies x \equiv -(k - 1) \pmod{p_{2n - 1} p_{2n}}
    \end{align}
    be true, since this defines a length \(k\) streak of consecutive non-prime power integers.
    Since the product of two primes is always coprime with any other product of two primes, we can use the Chinese remainder theorem to verify that there exists a solution \(x\) that satisfies the above.
\end{proof}

\section{Fermat's Little Theorem}

\begin{theorem}
    \(42 | n^7 - n, n \in \N\)
\end{theorem}
\begin{proof}
    \begin{align}
        42 | n^7 - n \implies n^7 - n &\equiv 0 \pmod{42} \\
        n^7 &\equiv n \pmod{6 \cdot 7}
    \end{align}
    \begin{lemma}
        If \(x \equiv a \pmod{n}\), with \(a < n\), then \(x \equiv a \pmod{b n}\), where \(b \in \N\), and \(x\) being unique.
    \end{lemma} \label{thm:5-flt}
    \begin{proof}
        \begin{align}
            x \equiv a \pmod{n} \implies x &= a + nk \\
            &\equiv a + bnk \pmod{n} \\
            &= a + (bn)k \equiv a \pmod{bn}
        \end{align}
    \end{proof}
    This means if \(n^7 \equiv n \pmod{7}\), which is guaranteed by the variant of Fermat's Little Theorem, then \(n^7 \equiv n \pmod{42}\) by \autoref{thm:5-flt}, where \(x = n^7, a = n, b = 6\).
\end{proof}

\section{A Taste of RSA}

\begin{theorem}
    \(a^{(p - 1)(q - 1) + 1} \equiv a \pmod{N}\), for primes \(p, q, N = pq, a \in \N\) such that \(\gcd(a, N) = 1\).
\end{theorem}
\begin{proof}
    Consider \(a^{(p - 1)(q - 1) + 1} \pmod{p}\) and \(a^{(p - 1)(q - 1) + 1} \pmod{q}\).
    By Fermat's Little Theorem,
    \begin{align}
        a^{(p - 1)(q - 1)} \cdot \cancel{a} &\equiv \cancel{a} \pmod{p} \\
        (a^{(p - 1)})^{(q - 1)} \equiv 1^{(q - 1)} &\equiv 1 \pmod{p}
    \end{align}
    By similar argument,
    \begin{align}
        a^{(p - 1)(q - 1)} \cdot \cancel{a} &\equiv \cancel{a} \pmod{q} \\
        (a^{(q - 1)})^{(p - 1)} \equiv 1^{(p - 1)} &\equiv 1 \pmod{q}
    \end{align}
    By construction, \(p, q\) are prime, and thus are coprime.
    Therefore, by the Chinese remainder theorem, there exists a unique solution such that
    \begin{equation}
        a^{(p - 1)(q - 1)} \equiv 1 \pmod{pq}
    \end{equation}
    Multiplying by \(a\) on both sides yields the original statement.
\end{proof}

\section{Sundry}

I did homework with:
\begin{itemize}
    \item Manny Fuentes \href{mailto:mfuentes@berkeley.edu}{mfuentes@berkeley.edu}
    \item Avery Perez \href{mailto:aj_perez@berkeley.edu}{aj\_perez@berkeley.edu}
\end{itemize}

\end{document}
