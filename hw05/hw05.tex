\documentclass{article}
\usepackage{amsmath, amssymb, amsfonts, amsthm}
\usepackage{cancel}
\usepackage[output-complex-root=j]{siunitx}
\usepackage[american, nooldvoltagedirection]{circuitikz}
\usepackage{bm}
\usepackage{listings}
\usepackage{graphicx}
\usepackage{fullpage}
\usepackage{hyperref}

\renewcommand{\thesection}{\arabic{section}}
\renewcommand{\thesubsection}{\thesection.\alph{subsection}}
\renewcommand{\thesubsubsection}{\thesubsection.\roman{subsubsection}}
\newcommand{\lemmaautorefname}{Lemma}

\newtheorem{theorem}{Theorem}
\newtheorem{lemma}{Lemma}

\newcommand{\N}{\mathbb{N}}
\newcommand{\Z}{\mathbb{Z}}
\newcommand{\Q}{\mathbb{Q}}
\newcommand{\R}{\mathbb{R}}
\renewcommand{\C}{\mathbb{C}}
\newcommand{\unit}[1]{\bm{\hat{#1}}}
\newcommand{\iprod}[2]{\left\langle #1, #2 \right\rangle}
\newcommand{\tpose}[1]{\left[#1\right]^{\! \top} \!\!}
\newcommand{\diff}[1]{\frac{d}{d #1}}

\lstset{
    language=Python,
    tabsize=4,
    basicstyle=\ttfamily,
    numbers=left,
    numberstyle=\ttfamily,
    keywordstyle=\color{blue},
    frame=single
}

\title{CS 70 HW05}
\author{Bryan Ngo}
\date{2020-09-28}

\begin{document}

\maketitle

\section{RSA with Just One Prime}

\subsection{}

\begin{theorem}
    RSA with one prime is cryptographically correct, or
    \begin{equation}
        x^{ed} \equiv x \pmod{p}
    \end{equation}
\end{theorem}
\begin{proof}
    By construction, we can represent \(ed\) as
    \begin{equation}
        ed = k(p - 1) + 1
    \end{equation}
    for some \(k \in \Z\).
    Then,
    \begin{align}
        x^{k(p - 1) + 1} &\equiv x \pmod{p} \\
        x^{k(p - 1)} &\equiv 1 \pmod{p} \\
        (x^{p - 1})^k \equiv 1^k &\equiv 1 \pmod{p}
    \end{align}
    where the last step follows from Fermat's Little Theorem, which is valid since \(\gcd(x, p) = 1\).
\end{proof}

\subsection{}

Eve can use the extended GCD algorithm to recover \(d\), essentially brute-forcing.
Since \(\gcd(e, p - 1) = 1\), the extended GCD is guaranteed to find the inverse of \(e \pmod{p - 1}\).
Since extended GCD has a worst-case runtime of \(O(n)\), where \(n\) is the number of bits, meaning that at worst Eve has to run the algorithm \(1024\) iterations.

\subsection{}

In order to recover \(x\), Eve would
\begin{enumerate}
    \item Calculate \(d\) through the calculation outlined in the previous problem.
    \item Simply perform the operation \(D(y)\) to recover \(x\).
\end{enumerate}
This takes constant time since Eve has presumably already recovered \(d\) and the modulo operation is constant time.

\subsection{}

Yes, this is possible because the main difficulty of RSA is that \(N = pq\) is infeasibly difficult to recover \(p, q\) from, since \(d, e\) are calculated modulo \((p - 1) (q - 1)\).
Since \(N = p\), we only need to subtract \(1\) in order to start regenerating the keys.
At most, as explained in previous problems, Eve only needs \(1024\) iterations of the extended GCD \emph{at most} to break "RSA", which is quite trivial.

\section{Squared RSA}

\subsection{}

\begin{theorem} \label{thm:2a}
    Given coprime \(a, p\) where \(p\) is prime,
    \begin{equation}
        a^{p (p - 1)} \equiv 1 \pmod{p^2}
    \end{equation}
\end{theorem}
\begin{proof}
    Consider the finite field \(\mathbb{F}_{p^2} = \{1, 2, \ldots, p^2 - 1\}\), \emph{excluding} every \(f \in \mathbb{F}_{p^2}\) such that \(\gcd(f, p^2) \neq 1\).
    This means that
    \begin{equation}
        S = \{1, 2, \ldots, p - 1, p + 1, \ldots, p^2 - 1\}
    \end{equation}
    Since \(\gcd(a, p) = 1\), the set \(S' = \{a \cdot s | s \in S\}\) is bijective to \(S\).
    Note that \(|S| = |S'| = p^2 - (p - 1) - 1 = p^2 - p\), since we have effectively excluded \(p - 1\) elements from \(\mathbb{F}_{p^2}\) when constructing \(S\), i.e. every \(k \in [1, p - 1]\).
    Then, consider the product of all elements in \(S\) and \(S'\) modulo \(p^2\),
    \begin{align}
        \prod_{i = 1}^{p^2 - p} a s_i &\equiv \prod_{i = 1}^{p^2 - p} s_i \pmod{p^2} \\
        a^{p^2 - p} \cancel{\prod_{i = 1}^{p^2 - p} s_i} &\equiv \cancel{\prod_{i = 1}^{p^2 - p} s_i} \pmod{p^2} \\
        a^{p (p - 1)} &\equiv 1 \pmod{p^2}
    \end{align}
\end{proof}

\subsection{}

\begin{theorem}
    The RSA scheme where \(N = p^2 q^2\) for prime \(p, q\) is correct; that is,
    \begin{equation}
        x^{ed} \equiv x \pmod{N}
    \end{equation}
    for \(x\) coprime to \(p, q\) and private and public keys \(e, d\).
\end{theorem}
\begin{proof}
    We can represent \(ed\) as
    \begin{equation}
        ed = k p(p - 1) q(q - 1) + 1
    \end{equation}
    for \(k \in \Z\).
    Thus,
    \begin{align}
        x^{k p(p - 1) q(q - 1) + 1} &\equiv x \pmod{p^2 q^2} \\
        \Rightarrow x (x^{k p(p - 1) q(q - 1)} - 1) &\equiv 0 \pmod{p^2 q^2}
    \end{align}
    To prove this, consider the lemma
    \begin{lemma} \label{thm:2b}
        \begin{equation}
            x (x^{k p(p - 1) q(q - 1)} - 1) \equiv 0 \pmod{p^2}
        \end{equation}
    \end{lemma}
    \begin{proof}
        \begin{align}
            x (x^{k p(p - 1) q(q - 1)} - 1) &\equiv 0 \pmod{p^2} \\
            x \left((x^{p(p - 1)})^{k q(q - 1)} - 1\right) &\equiv x (1^{k q(q - 1)} - 1) \equiv 0 \pmod{p^2} \\
            &\equiv 0 \pmod{p^2}
        \end{align}
        where we use \autoref{thm:2a} to reduce the system.
    \end{proof}
    Using \autoref{thm:2b}, we prove the case for \(p^2\).
    Without loss of generality, we can prove a similar case for \(q^2\).
    Since the term is divisible by both \(p^2, q^2\), it must also be divisible by \(N = p^2 q^2\).
\end{proof}

\section{The CRT and Lagrange Interpolation}

\subsection{}

\begin{theorem}
    Given the modular system of equations
    \begin{align}
        x &\equiv a_1 \pmod{n_1} \\
        x &\equiv a_2 \pmod{n_2}
    \end{align}
    there exists a solution \(x_1\) for \(a_1 = 1, a_2 = 0\).
\end{theorem}
\begin{proof}
    Since \(a_2 \equiv 0 \pmod{n_2}\), \(x_1 = k_2 n_2\) for some \(k_2 \in \Z\).
    Then, since \(a_1 \equiv 1 \pmod{n_1}\),
    \begin{equation}
        k_2 n_2 \equiv 1 \pmod{n_1}
    \end{equation}
    Then, by \href{https://www.eecs70.org/static/notes/n6.pdf}{Theorem 6.2}, since \(\gcd(n_2, n_1) = 1\) by definition, there \emph{must} exist some \(k_2\) satisfying this.
    Therefore, there exists some \(x_1\) satisfying the equation.
\end{proof}
\begin{theorem}
    Given the modular system of equations
    \begin{align}
        x &\equiv a_1 \pmod{n_1} \\
        x &\equiv a_2 \pmod{n_2}
    \end{align}
    there exists a solution \(x_2\) for \(a_1 = 0, a_2 = 2\).
\end{theorem}
\begin{proof}
    Since \(a_1 \equiv 0 \pmod{n_1}\), \(x_1 = k_1 n_1\) for some \(k_1 \in \Z\).
    Then, since \(a_2 \equiv 1 \pmod{n_2}\),
    \begin{equation}
        k_1 n_1 \equiv 1 \pmod{n_2}
    \end{equation}
    Then, by \href{https://www.eecs70.org/static/notes/n6.pdf}{Theorem 6.2}, since \(\gcd(n_2, n_1) = 1\) by definition, there \emph{must} exist some \(k_1\) satisfying this.
    Therefore, there exists some \(x_2\) satisfying the equation.
\end{proof}

\subsection{}

\begin{theorem}
    Given the modular system of equations
    \begin{align}
        x &\equiv a_1 \pmod{n_1} \\
        x &\equiv a_2 \pmod{n_2}
    \end{align}
    there exists a solution \(x\) for any \(a_1, a_2\) which are all equivalent modulo \(n_1 n_2\).
\end{theorem}
\begin{proof}
    Since we've proven that there exists some \(k \in \Z\) solving
    \begin{equation}
        k n_2 \equiv 1 \pmod{n_1}
    \end{equation}
    we can simply multiply by any natural \(a \in [1, n_1 - 1]\) to obtain any solution modulo \(n_1\).
    Without loss of generality, this also applies modulo \(n_2\).
    Then, our solution \(x\) is
    \begin{equation}
        x = a_2 k_1 n_1 + a_1 k_2 n_2
    \end{equation}
    which is identically equivalent up to modulo \(n_1 n_2\) since it is divisible by the individual factors \(n_1, n_2\), so the sum also has a unique solution.
\end{proof}

\subsection{}

\begin{theorem}
    Given the modular system of equations
    \begin{align}
        x &\equiv a_1 \pmod{n_1} \\
        x &\equiv a_2 \pmod{n_2} \\
        &\vdots \\
        x &\equiv a_k \pmod{n_k}
    \end{align}
    there exists a solution \(x\) for any \(a_{1, 2, \ldots, k}\) which is unique modulo \(\prod_{i = 1}^k n_i\).
\end{theorem}
\begin{proof}
    Base case: See the above subproblem.
    Inductive hypothesis: CRT holds for \(i\) equations.
    For \(i + 1\) equations, since CRT hold for the previous \(i\) equations, then
    \begin{equation}
        k_i \prod_{j = 1, j \neq i + 1}^i n_j \equiv 1 \pmod{n_{i + 1}}
    \end{equation}
    Then, since by definition \(\gcd(n_i, n_{i + 1}) = 1\), there must exist some \(k_i\) that is the modular inverse.
    Then, we can also multiply by any \(a_j \in [1, n_{i + 1} - 1]\) to obtain any distinct number modulo \(n_{i + 1}\).
    So our solution is
    \begin{equation}
        x = \sum_{j = 1, j \neq i}^{i + 1} a_j k_j n_j
    \end{equation}
    Since there is a bijective mapping between unknowns and moduli, the solution is unique.
\end{proof}

\subsection{}

\begin{theorem}
    The modular system of equations
    \begin{align}
        p(x) &\equiv y_1 \pmod{(x - x_1)} \\
        p(x) &\equiv y_2 \pmod{(x - x_2)} \\
        &\vdots \\
        p(x) &\equiv y_k \pmod{(x - x_k)}
    \end{align}
    has a unique solution modulo \(\prod_{i = 1}^k (x - x_i)\).
\end{theorem}
\begin{proof}
    Since \(\gcd(x_i, x_j) = 1\), \(\gcd(x - x_i, x - x_j) = 1\).
    This means that all the moduli are coprime, since the only degree one factor they could share is themselves.
    Thus, we can use the Chinese remainder theorem to prove that there exists some \(p(x)\) modulo the product of all the basis binomials.
\end{proof}
We can invoke Lagrange interpolation by solving the system.
For some modulus \(x - x_i\), we construct some \(\Delta_i(x)\) such that
\begin{equation}
    \Delta_i(x) = a_i \prod_{j = 1, j \neq i}^i (x - x_j) \equiv y_i \pmod{(x - x_i)}
\end{equation}
We can simply find the modular inverse of \(\prod_{j = 1, j \neq i}^i (x - x_j)\) modulo \(x - x_i\), the multiply by \(y_i\) to obtain \(\Delta_i(x)\).
Our solution \(p(x)\) is thus
\begin{equation}
    p(x) = \sum_{i = 1}^k y_i \Delta_i(k)
\end{equation}

\section{Polynomials in Fields}

\subsection{}

\begin{theorem}
    \begin{equation}
        P_n(7) \equiv 0 \pmod{19}
    \end{equation}
    for some \(n \in \N\) and \(P_n(x)\) is defined as
    \begin{equation}
        P_n(x) =
        \begin{cases}
            x + 12 & n = 0 \\
            x^2 - 5x + 5 & n = 1 \\
            x P_{n - 2}(x) - P_{n - 1}(x) & \textnormal{otherwise}
        \end{cases}
    \end{equation}
\end{theorem}
\begin{proof}
    Base case:
    \begin{align}
        P_0(7) = 7 + 12 = 19 &\equiv 0 \pmod{19} \\
        P_1(7) = 49 - 35 + 5 = 19 &\equiv 0 \pmod{19}
    \end{align}
    Inductive hypothesis: \(P_k(7) \equiv 0 \pmod{19}\).
    \begin{align}
        P_{k + 1}(7) = 7P_{k - 1} - P_k(7)
    \end{align}
    By strong induction, all cases between the base case and the inductive step satisfy the hypothesis, so we have
    \begin{equation}
        7P_{k - 1} - P_k(7) = 7 \cdot 19k_1 - 19k_2 = 19(7k_1 - k_2) \equiv 0 \pmod{19}
    \end{equation}
    for some \(k_1, k_2 \in \Z\).
\end{proof}

\subsection{}

\begin{theorem}
    For all prime \(q\), if \(P_{2017}(x) \not\equiv 0 \pmod{q}\), then \(P_{2017}(x)\) has at most \(2017\) roots modulo \(q\).
\end{theorem}
\begin{proof}
    Using induction on the degree of \(P_n(x)\),
    \begin{lemma}
        \begin{equation}
            \deg(P_n(x)) \leqslant n + 1
        \end{equation}
    \end{lemma}
    \begin{proof}
        Base case: \(\deg(P_0(x)) = 1\), \(\deg(P_1(x)) = 2\).
        Inductive hypothesis: \(\deg(P_k(x)) = k + 1\).
        \begin{equation}
            P_{k + 1}(x) = x P_{k - 1}(x) - P_k(x)
        \end{equation}
        By strong induction,
        \begin{align}
            \deg(x P_{k - 1}(x) - P_k(x)) &\leqslant \max\{\deg(x P_{k - 1}(x)), \deg(P_k(x))\} \\
            \deg(x P_{k - 1}(x)) &= k + 1 \\
            \Rightarrow \deg(x P_{k - 1}(x) - P_k(x)) &\leqslant \max\{k + 1, k + 1\} = k + 2
        \end{align}
    \end{proof}
    By the fundamental theorem of algebra, the theorem holds.
\end{proof}

\section{Secrets in the United Nations}

\subsection{}

The number of keys required is as follows:
\begin{itemize}
    \item \(193\) keys for each country.
    \item \(193 - 55 = 138\) keys for the Secretary General since only he and \(55\) other countries are required to open the vault.
\end{itemize}
which means that there will be a total of \(331\) unique keys distributed.
By the properties of the Shamir Secret Sharing Scheme outlined in \href{https://www.eecs70.org/static/notes/n8.pdf}{Note 8}, we require a polynomial of degree \(192\) because only \(193\) unique keys are required to open the vault.

\subsection{}

The only modification we need to make is after generating our keys for the degree \(192\) polynomial for each country, encode their secret \(s = (i, P(i))\) as the solution to a new degree \(11\) polynomial, then distribute \emph{those} keys to each representative.

\section{Secret Sharing with Spies}

The scheme is as follows, where \(s\) is the number of spies:
\begin{itemize}
    \item The number of keys required to decrypt the secret is \(s + 1 = 4\), so the spies cannot decrypt it themselves.
    \item If we divide the \(M\) troops into groups of \(4\), assume the worst case scenario where one spy is in each group, meaning they sabotage \(3\) groups
    Then, we implement a majority vote system where if \(s + 1\) groups decrypt the secret, then the secret is decrypted.
    This means that there must be \(7\) groups of \(4\) at minimum, so \(M \geqslant 28\).
    For any \(M < 28\), you run the risk of the spies sabotaging at least half of the groups, failing to decrypt the secret.
    For example, if \(M = 24\), the spies could sabotage at most \(\frac{1}{2}\) of the groups.
\end{itemize}

\section{Sundry}

I worked with
\begin{itemize}
    \item Victor Zhang \href{mailto:vzhang6@berkeley.edu}{vzhang6@berkeley.edu}
\end{itemize}

\end{document}
