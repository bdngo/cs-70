\documentclass{article}
\usepackage{amsmath, amssymb, amsfonts, amsthm}
\usepackage{cancel}
\usepackage[output-complex-root=j]{siunitx}
\usepackage[american, nooldvoltagedirection]{circuitikz}
\usepackage{bm}
\usepackage{listings}
\usepackage{graphicx}
\usepackage{fullpage}
\usepackage{hyperref}

\renewcommand{\thesection}{\arabic{section}}
\renewcommand{\thesubsection}{\thesection.\alph{subsection}}
\renewcommand{\thesubsubsection}{\thesubsection.\roman{subsubsection}}
\newcommand{\lemmaautorefname}{Lemma}

\newtheorem{theorem}{Theorem}
\newtheorem{lemma}{Lemma}

\newcommand{\N}{\mathbb{N}}
\newcommand{\Z}{\mathbb{Z}}
\newcommand{\Q}{\mathbb{Q}}
\newcommand{\R}{\mathbb{R}}
\renewcommand{\C}{\mathbb{C}}
\newcommand{\unit}[1]{\bm{\hat{#1}}}
\newcommand{\iprod}[2]{\left\langle #1, #2 \right\rangle}
\newcommand{\tpose}[1]{\left[#1\right]^{\! \top} \!\!}
\newcommand{\diff}[1]{\frac{d}{d #1}}

\lstset{
    language=Python,
    tabsize=4,
    basicstyle=\ttfamily,
    numbers=left,
    numberstyle=\ttfamily,
    keywordstyle=\color{blue},
    frame=single
}

\title{CS 70 HW05}
\author{Bryan Ngo}
\date{2020-09-28}

\begin{document}

\maketitle

\section{RSA with Just One Prime}

\subsection{}

\begin{theorem}
    RSA with one prime is cryptographically correct, or
    \begin{equation}
        x^{ed} \equiv x \pmod{p}
    \end{equation}
\end{theorem}
\begin{proof}
    By construction, we can represent \(ed\) as
    \begin{equation}
        ed = k(p - 1) + 1
    \end{equation}
    for some \(k \in \Z\).
    Then,
    \begin{align}
        x^{k(p - 1) + 1} &\equiv x \pmod{p} \\
        x^{k(p - 1)} &\equiv 1 \pmod{p} \\
        (x^{p - 1})^k \equiv 1^k &\equiv 1 \pmod{p}
    \end{align}
    where the last step follows from Fermat's Little Theorem, which is valid since \(\gcd(x, p) = 1\).
\end{proof}

\subsection{}

Eve can use the extended GCD algorithm to recover \(d\).
Since \(\gcd(e, p - 1) = 1\), the extended GCD is guaranteed to find the inverse of \(e \pmod{p - 1}\).
Since extended GCD has a worst-case runtime of \(O(n)\), where \(n\) is the number of bits, meaning that at worst Eve has to run the algorithm \(1024\) iterations.

\subsection{}

In order to recover \(x\), Eve would
\begin{enumerate}
    \item Calculate \(d\) through the calculation outlined in the previous problem.
    \item Simply perform the operation \(D(y)\) to recover \(x\).
\end{enumerate}

\subsection{}

Yes, this is possible because the main difficulty of RSA is that \(N = pq\) is infeasibly difficult to recover \(p, q\) from, since \(d, e\) are calculated modulo \((p - 1) (q - 1)\).
Since \(N = p\), we only need to subtract \(1\) in order to start regenerating the keys.
At most, as explained in previous problems, Eve only needs \(1024\) iterations of the extended GCD \emph{at most} to break "RSA", which is quite trivial.

\section{Squared RSA}

\subsection{}

\begin{theorem} \label{thm:2a}
    Given coprime \(a, p\) where \(p\) is prime,
    \begin{equation}
        a^{p (p - 1)} \equiv 1 \pmod{p^2}
    \end{equation}
\end{theorem}
\begin{proof}
    Consider the finite field \(\mathbb{F}_{p^2} = \{1, 2, \ldots, p^2 - 1\}\), \emph{excluding} every \(f \in \mathbb{F}_{p^2}\) such that \(\gcd(f, p^2) \neq 1\).
    This means that
    \begin{equation}
        S = \{1, 2, \ldots, p - 1, p + 1, \ldots, p^2 - 1\}
    \end{equation}
    Since \(\gcd(a, p) = 1\), the set \(S' = \{a \cdot s | s \in S\}\) is bijective to \(S\).
    Note that \(|S| = |S'| = p^2 - (p - 1) - 1 = p^2 - p\), since we have effectively excluded \(p - 1\) elements from \(\mathbb{F}_{p^2}\) when constructing \(S\), i.e. every \(k \in [1, p - 1]\).
    Then, consider the product of all elements in \(S\) and \(S'\) modulo \(p^2\),
    \begin{align}
        \prod_{i = 1}^{p^2 - p} a s_i &\equiv \prod_{i = 1}^{p^2 - p} s_i \pmod{p^2} \\
        a^{p^2 - p} \cancel{\prod_{i = 1}^{p^2 - p} s_i} &\equiv \cancel{\prod_{i = 1}^{p^2 - p} s_i} \pmod{p^2} \\
        a^{p (p - 1)} &\equiv 1 \pmod{p^2}
    \end{align}
\end{proof}

\subsection{}

\begin{theorem}
    The RSA scheme where \(N = p^2 q^2\) for prime \(p, q\) is correct; that is,
    \begin{equation}
        x^{ed} \equiv x \pmod{N}
    \end{equation}
    for \(x\) coprime to \(p, q\) and private and public keys \(e, d\).
\end{theorem}
\begin{proof}
    We can represent \(ed\) as
    \begin{equation}
        ed = k p(p - 1) q(q - 1) + 1
    \end{equation}
    for \(k \in \Z\).
    Thus,
    \begin{align}
        x^{k p(p - 1) q(q - 1) + 1} &\equiv x \pmod{p^2 q^2} \\
        \Rightarrow x (x^{k p(p - 1) q(q - 1)} - 1) &\equiv 0 \pmod{p^2 q^2}
    \end{align}
    To prove this, consider the lemma
    \begin{lemma} \label{thm:2b}
        \begin{equation}
            x (x^{k p(p - 1) q(q - 1)} - 1) \equiv 0 \pmod{p^2}
        \end{equation}
    \end{lemma}
    \begin{proof}
        \begin{align}
            x (x^{k p(p - 1) q(q - 1)} - 1) &\equiv 0 \pmod{p^2} \\
            x \left((x^{p(p - 1)})^{k q(q - 1)} - 1\right) &\equiv x (1^{k q(q - 1)} - 1) \equiv 0 \pmod{p^2} \\
            &\equiv 0 \pmod{p^2}
        \end{align}
        where we use \autoref{thm:2a} to reduce the system.
    \end{proof}
    Using \autoref{thm:2b}, we prove the case for \(p^2\).
    Without loss of generality, we can prove a similar case for \(q^2\).
    Since the term is divisible by both \(p^2, q^2\), it must also be divisible by \(N = p^2 q^2\).
\end{proof}

\section{The CRT and Lagrage Interpolation}

\subsection{}

\begin{theorem}
    Given the modular system of equations
    \begin{align}
        x &\equiv a_1 \pmod{n_1} \\
        x &\equiv a_2 \pmod{n_2}
    \end{align}
    there exists a solution \(x_1\) for \(a_1 = 1, a_2 = 0\).
\end{theorem}

\end{document}
