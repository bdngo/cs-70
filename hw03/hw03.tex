\documentclass{article}
\usepackage{amsmath, amssymb, amsfonts, amsthm}
\usepackage{cancel}
\usepackage[output-complex-root=j]{siunitx}
\usepackage[american, nooldvoltagedirection]{circuitikz}
\usepackage{bm}
\usepackage{listings}
\usepackage{graphicx}
\usepackage{fullpage}
\usepackage{hyperref}

\renewcommand{\thesection}{\arabic{section}}
\renewcommand{\thesubsection}{\thesection.\alph{subsection}}
\renewcommand{\thesubsubsection}{\thesubsection.\roman{subsubsection}}
\renewcommand{\labelenumi}{\alph{enumi}.}

\newtheorem{theorem}{Theorem}

\newcommand{\N}{\mathbb{N}}
\newcommand{\Z}{\mathbb{Z}}
\newcommand{\Q}{\mathbb{Q}}
\newcommand{\R}{\mathbb{R}}
\renewcommand{\C}{\mathbb{C}}
\newcommand{\unit}[1]{\bm{\hat{#1}}}
\newcommand{\iprod}[2]{\left\langle #1, #2 \right\rangle}
\newcommand{\tpose}[1]{\left[#1\right]^{\! \top} \!\!}
\newcommand{\diff}[1]{\frac{d}{d #1}}

\lstset{
    language=Python,
    tabsize=4,
    basicstyle=\ttfamily,
    numbers=left,
    numberstyle=\ttfamily,
    keywordstyle=\color{blue},
    frame=single
}

\title{CS 70 HW03}
\author{Bryan Ngo}
\date{2020-09-14}

\begin{document}

\maketitle

\section{Graph Basics}

\begin{enumerate}
    \item
    \begin{align}
        V &= \{A, B, C, D, E, F, G, H\} \\
        E &= \{(A, F), (A, B), (B, C), (B, E), (C, D), (D, H), (D, B), (E, D), (E, G), (G, F), (F, G), (F, E), (H, G)\}
    \end{align}
    \item \(G\) has the highest in-degree of \(3\), and \(A\) has the lowest in-degree of \(0\).
    Vertices with an in-degree of \(1\) are \(\{C, H\}\).
    Vertices with an in-degree of \(2\) are \(\{B, D, E, F\}\).
    \item The paths from \(B\) to \(F\) are \(\{(B, E, G, F), (B, C, D, H, G, F)\}\).
    The path \((B, E, G, F)\) is shortest.
    \item i and ii are cycles.
    \item i, ii, iii, and vi are walks.
    \item iii is a tour.
    \item False
    \item True
    \item False
\end{enumerate}

\section{Binary Trees}

\subsection{}

\begin{theorem} \label{thm:2a-1}
    Let \(T\) be a binary tree with \(h(T) > 0\), where \(h\) is the height of the tree.
    Then, let \(r\) be the root with left and right branches \(u, v\), respectively.
    Removing \(r\) results in two new binary trees \(L, R\) with respective roots \(u, v\).
\end{theorem}
\begin{proof}
    Removing \(r\) reduces the degree of both \(u, v\) by one, so they both now have degree \(2\).
    Thus, by our definition of a binary tree, they form roots.
\end{proof}

\begin{theorem}
    \begin{equation}
        h(T) = \max(h(L), h(R)) + 1
    \end{equation}
    where \(T, L, R\) are as defined in \autoref{thm:2a-1}.
\end{theorem}

\end{document}
