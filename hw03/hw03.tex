\documentclass{article}
\usepackage{amsmath, amssymb, amsfonts, amsthm}
\usepackage{cancel}
\usepackage[output-complex-root=j]{siunitx}
\usepackage[american, nooldvoltagedirection]{circuitikz}
\usepackage{bm}
\usepackage{listings}
\usepackage{graphicx}
\usepackage{fullpage}
\usepackage{hyperref}

\renewcommand{\thesection}{\arabic{section}}
\renewcommand{\thesubsection}{\thesection.\alph{subsection}}
\renewcommand{\thesubsubsection}{\thesubsection.\roman{subsubsection}}
\renewcommand{\labelenumi}{\alph{enumi}.}

\newtheorem{theorem}{Theorem}
\newtheorem{lemma}{Lemma}

\newcommand{\N}{\mathbb{N}}
\newcommand{\Z}{\mathbb{Z}}
\newcommand{\Q}{\mathbb{Q}}
\newcommand{\R}{\mathbb{R}}
\renewcommand{\C}{\mathbb{C}}
\newcommand{\unit}[1]{\bm{\hat{#1}}}
\newcommand{\iprod}[2]{\left\langle #1, #2 \right\rangle}
\newcommand{\tpose}[1]{\left[#1\right]^{\! \top} \!\!}
\newcommand{\diff}[1]{\frac{d}{d #1}}

\lstset{
    language=Python,
    tabsize=4,
    basicstyle=\ttfamily,
    numbers=left,
    numberstyle=\ttfamily,
    keywordstyle=\color{blue},
    frame=single
}

\title{CS 70 HW03}
\author{Bryan Ngo}
\date{2020-09-14}

\begin{document}

\maketitle

\section{Graph Basics}

\begin{enumerate}
    \item
    \begin{align}
        V &= \{A, B, C, D, E, F, G, H\} \\
        E &= \{(A, F), (A, B), (B, C), (B, E), (C, D), (D, H), (D, B), (E, D), (E, G), (G, F), (F, G), (F, E), (H, G)\}
    \end{align}
    \item \(G\) has the highest in-degree of \(3\), and \(A\) has the lowest in-degree of \(0\).
    Vertices with an in-degree of \(1\) are \(\{C, H\}\).
    Vertices with an in-degree of \(2\) are \(\{B, D, E, F\}\).
    \item The paths from \(B\) to \(F\) are \(\{(B, E, G, F), (B, C, D, H, G, F), (B, E, D, H, G, F)\}\).
    The path \((B, E, G, F)\) is shortest.
    \item i and ii are cycles.
    \item i, ii, iii, and vi are walks.
    \item iii is a tour.
    \item False
    \item True
    \item False
\end{enumerate}

\section{Binary Trees}

\subsection{}

\begin{theorem} \label{thm:2a-1}
    Let \(T\) be a binary tree with \(h(T) > 0\), where \(h\) is the height function.
    Then, let \(r\) be the root with left and right branches \(u, v\), respectively.
    Removing \(r\) results in two new binary trees \(L, R\) with respective roots \(u, v\).
\end{theorem}
\begin{proof}
    Removing \(r\) reduces the degree of both \(u, v\) by one, so they both now have degree \(2\).
    Thus, by our definition of a binary tree, they form roots.
\end{proof}

\begin{theorem}
    \begin{equation}
        h(T) = \max(h(L), h(R)) + 1
    \end{equation}
    where \(T, L, R\) are as defined in \autoref{thm:2a-1}.
\end{theorem}
\begin{proof}
    When removing \(r\), the height of the tree is reduced by \(1\).
    Then, either \(L, R\) will terminate with a greater or equal height.
    Reversing this process leads to the equation defined in the theorem.
\end{proof}

\subsection{}

\begin{theorem} \label{thm:2c}
    The maximum number of vertices in a binary tree with height \(h\) is \(2^{h + 1} - 1\).
\end{theorem}
\begin{proof}
    The series describing the sum of vertices at each height is
    \begin{equation}
        v(h) = \sum_{i = 0}^h 2^k
    \end{equation}
    which can be proved inductively.
    Then, using the formula for the sum of a finite geometric series,
    \begin{equation}
        v(h) = \frac{1 - 2^{h + 1}}{1 - 2} = -(1 - 2^{h + 1}) = 2^{h + 1} - 1
    \end{equation}
\end{proof}

\subsection{}

\begin{theorem}
    A binary tree with \(n\) leaves has \(2n - 1\) vertices.
\end{theorem}
\begin{proof}
    Observe that \(h = \log_2(n)\).
    This follows from the fact that the number of leaves doubles each time the height increases by 1; this means that \(n = 2^h\).
    Then, using \autoref{thm:2c},
    \begin{equation}
        v(n) = 2 \cdot 2^{\log_2(n)} - 1 = 2n - 1
    \end{equation}
\end{proof}

\section{Proofs in Graphs}

\subsection{}

\begin{theorem}
    Given a connected directed graph with \(n \geqslant 2\) vertices, a path connecting two vertices is no longer than \(2\).
\end{theorem}
\begin{proof}
    Base case: \(n = 2\) means there is a path of length \(1\) connecting the two vertices.
    Inductive step: A connected directed graph with \(k\) vertices has a path no longer than \(2\).
    Suppose a graph with \(k + 1\) vertices.
    Consider a path with a "starting", "middle", and "end" vertex.
    Then, in the case for which there is a path of two, it can be actualized as two paths of length one.
    There are several cases to consider:
    \begin{itemize}
        \item If the \((k + 1)\)th vertex is connected to the starting vertex, then a path is formed between the middle and starting vertex to the new vertex.
        \item If it is connected to the middle vertex, either it has a length two path from the end vertex or the starting vertex, depending on the direction.
        \item If it is connected to the end vertex, then this is simply the first case with the directions flipped.
    \end{itemize}
    If the new vertex only has an in-degree of \(0\), then it is the starting point of the rest of the vertices, which reduces to the inductive hypothesis.
    If the new vertex only has an out-degree of \(0\), then it is the destination of the rest of the vertices.
\end{proof}

\subsection{}

\begin{theorem}
    Given a connected graph \(G\) with \(n\) vertices with \(2m\) vertices with odd degree, there are exactly \(m\) Eulerian tours.
\end{theorem}
\begin{proof}
    The proof is left as an exercise to the reader.
\end{proof}

\section{Planarity}

\subsection{}

\begin{theorem}
    The \(K_{3, 3}\) graph is nonplanar.
\end{theorem}
\begin{proof}
    Observe that \(K_{3, 3}\) has \(v = 6\) and \(e = 9\).
    By definition of being bipartite, there are no 3-cycle from one color to another.
    This means that each face emcompasses at least 4 edges, or \(s_i \geqslant 4\), so
    \begin{equation}
        4f \leqslant 2e
    \end{equation}
    Solving for \(f\) and plugging into the Euler characteristic for a sphere,
    \begin{align}
        v - e + f &= 2 \\
        v - e + \frac{1}{2}e &\geqslant 2 \\
        \Rightarrow v - \frac{1}{2}e &\geqslant 2 \\
        \Rightarrow e &\leqslant 2v - 4
    \end{align}
    Plugging in the edges and vertices for \(K_{3, 3}\), we see that \(9 \nleqslant 2(6) - 4 = 8\), proving nonplanarity.
\end{proof}

\subsection{}

\begin{theorem}
    Let \(T\) be the property that for every distinct vertices \(v_1, v_2, v_3\), there are at least \(2\) edges among them.
    Then, graph \(G\) with \(v \geqslant 7\) vertices is nonplanar.
\end{theorem}
\begin{proof}
    Assume that given \(v_1, v_2, v_3\) in \(v \geqslant 7\), there are no more than 2 edges among them.
    The rest of the proof is left as an exercise to the reader.
\end{proof}

\section{Always, Sometimes, or Never}

\begin{enumerate}
    \item \(OG\) must be planar, the four-color theorem only applies to planar graphs.
    \item \(OG\) must be non-planar.
    By the contrapositive of the five-color theorem (Theorem 5.4 in Note 5), since 7 colors are required, it \emph{cannot} be colored with 5, so it must be nonplanar.
    \item \(OG\) can be either.
    Planar: \(K_n\) where \(n < 5\).
    Nonplanar: \(K_{3, 3}: 9 \leqslant 3(6) - 6 = 12\).
    \item \(OG\) must be planar, since \(K_5\) and \(K_{3, 3}\) have maximum degrees of \(4\) and \(3\), respectively.
    By Kuratowski's theorem (Theorem 5.3 in Note 5), all connected nonplanar graphs have max degree greater than \(2\).
    \item \(OG\) must be planar, Kuratowski's theorem still applies as above, the only other nuance being that \(OG\) need not be connected.
    However, since \(K_5\) and \(K_{3, 3}\) are connected and by Kuratowski's theorem, all nonplanar graphs are derived from them, \(OG\) must be planar.
\end{enumerate}

\section{Touring Hypercube}

\subsection{}

\begin{theorem}
    A hypercube has an Eulerian tour if and only if \(n\) is even.
\end{theorem}
\begin{proof}
    To prove this we need only prove that the degree of a hypercube is equal to the dimension of said hypercube.
    \begin{lemma}
        For a hypercube with dimension \(k\), its degree \(d = k\).
    \end{lemma}
    \begin{proof}
        Base case: \(d = 0\), \(n = 0\) as well since it is simply a single vertex.
        Inductive hypothesis: \(d = k\) for a \(k\)-degree cube.
        Suppose a hypercube with dimension \(k + 1\).
        Then, by definition of a hypercube, then each vertex in the \(k\)-dimension subcube has exactly one edge mapping to the other \(k\)-dimension subcube.
        Therefore, the degree of each vertex has also increased by \(1\) as well.
    \end{proof}
    Since \(d = k\), then for even degree/dimension, by Euler's theorem (Theorem 5.1 in Note 5), only even-dimensioned hypercubes have an Eulerian tour.
\end{proof}

\subsection{}

\begin{theorem}
    Every hypercube has a Hamiltonian tour.
\end{theorem}
\begin{proof}
    Base case: \(n = 1\), simply cross an edge from \(0 \to 1\).
    Inductive hypothesis: A \(k\)-dimensional hypercube has a Hamiltonian tour.
    For \(k + 1\), we find the Hamiltonian tour for the \(k\)-dimensional subcube, then cross the edge (flipping a guaranteed untouched bit), then reverse the directions.
    In other words, in pseudocode:
    \begin{lstlisting}
def hamiltonian_tour(n):
    if n == 1:
        return (0, 1)
    return hamiltonian(n - 1) + n - hamiltonian(n - 1)
    \end{lstlisting}
\end{proof}

\section{Sundry}

I did homework with:
\begin{itemize}
    \item Manny Fuentes \href{mailto:mfuentes@berkeley.edu}{mfuentes@berkeley.edu}
    \item Avery Perez \href{mailto:aj_perez@berkeley.edu}{aj\_perez@berkeley.edu}
    \item Alex Chuang \href{mailto:alexc0413@berkeley.edu}{alexc0413@berkeley.edu}
\end{itemize}

\end{document}
