\documentclass{article}
\usepackage{amsmath, amssymb, amsfonts, amsthm}
\usepackage{cancel}
\usepackage[binary-units=true, output-complex-root=j]{siunitx}
\usepackage[american, nooldvoltagedirection]{circuitikz}
\usepackage{bm}
\usepackage{listings}
\usepackage{graphicx}
\usepackage{fullpage}
\usepackage{hyperref}

\renewcommand{\thesection}{\arabic{section}}
\renewcommand{\thesubsection}{\thesection.\alph{subsection}}
\renewcommand{\thesubsubsection}{\thesubsection.\roman{subsubsection}}
\newcommand{\lemmaautorefname}{Lemma}
\renewcommand{\labelenumi}{\alph{enumi}.}
\renewcommand{\labelenumii}{\roman{enumii}.}

\newtheorem{theorem}{Theorem}
\newtheorem{lemma}{Lemma}

\newcommand{\N}{\mathbb{N}}
\newcommand{\Z}{\mathbb{Z}}
\newcommand{\Q}{\mathbb{Q}}
\newcommand{\R}{\mathbb{R}}
\renewcommand{\C}{\mathbb{C}}
\newcommand{\unit}[1]{\bm{\hat{#1}}}
\newcommand{\iprod}[2]{\left\langle #1, #2 \right\rangle}
\newcommand{\tpose}[1]{\left[#1\right]^{\! \top} \!\!}
\newcommand{\diff}[1]{\frac{d}{d #1}}

\lstset{
    language=Python,
    tabsize=4,
    basicstyle=\ttfamily,
    numbers=left,
    numberstyle=\ttfamily,
    keywordstyle=\color{blue},
    frame=single
}

\title{CS 70 HW06}
\author{Bryan Ngo}
\date{2020-10-09}

\begin{document}

\maketitle

\section{Exam Policy and Practice}

\begin{enumerate}
    \item \begin{enumerate}
        \item \SI{20}{\minute}
        \item You are allowed to record locally, with about \(\SI{220}{\mega\byte\per\hour} \cdot \SI{2}{\hour} = \SI{440}{\mega\byte}\) required.
        \item Either fill out the Exam Disruption Form or make a private post on Piazza.
    \end{enumerate}
    \item The magic word is \textbf{Signature}.
\end{enumerate}

\section{Message is Too Noisy}

\subsection{}

\begin{align}
    \Delta_0(x) &= \frac{(x - 1) (x - 2)}{(0 - 1) (0 - 2)} = \frac{(x - 1) (x - 2)}{2} \equiv 9 (x - 1) (x - 2) \pmod{17} \\
    \Delta_1(x) &= \frac{x (x - 2)}{1 (1 - 2)} \\
    \Delta_2(x) &= \frac{x (x - 1)}{2 (2 - 1)} \\
    P(x) &= 0 \Delta_0(x) + 1 \Delta_1(x) + 4 \Delta_2(x) = -x^2 + 2x + 2x^2 - 2x = x^2
\end{align}

\subsection{}

The total packets are \((0, 1, 4, 9, 0)\).
Then, our \(Q(x) = a_0 + a_1 x + a_2 x^2 + a_3 x^3\), so
\begin{align}
    a_0 &\equiv 0 \pmod{17} \\
    a_0 + a_1 + a_2 + a_3 + e_0 &\equiv 1 \pmod{17} \\
    a_0 + 2 a_1 + 4 a_2 + 8 a_3 + 4 e_0 &\equiv 8 \pmod{17} \\
    a_0 + 3 a_1 + 9 a_2 + 27 a_3 + 9 e_0 &\equiv 27 \pmod{17} \\
    a_0 + 4 a_1 + 16 a_2 + 64 a_3 &\equiv 0 \pmod{17} \\
    \Rightarrow
    \begin{bmatrix}
        1 & 0 & 0 & 0 & 0 \\
        1 & 1 & 1 & 1 & 1 \\
        1 & 2 & 4 & 8 & 4 \\
        1 & 3 & 9 & 27 & 9 \\
        1 & 4 & 16 & 64 & 0
    \end{bmatrix}
    \begin{bmatrix}
        a_0 \\
        a_1 \\
        a_2 \\
        a_3 \\
        e_0
    \end{bmatrix}
    &\equiv
    \begin{bmatrix}
        0 \\
        1 \\
        8 \\
        27 \\
        0
    \end{bmatrix} \pmod{17} \\
    \Rightarrow
    \begin{bmatrix}
        a_0 \\
        a_1 \\
        a_2 \\
        a_3 \\
        e_0
    \end{bmatrix}
    =
    \begin{bmatrix}
        0 \\
        0 \\
        -4
        1 \\
        4
    \end{bmatrix}
\end{align}
So our solutions are
\begin{align}
    Q(x) &= x^3 - 4x^2 = x^2 (x - 4) \\
    E(x) &= x - 4 \\
    P(x) &= \frac{Q(x)}{E(x)} = x^2
\end{align}

\subsection{}

The total packets are \((0, 1, 4, 9, 0)\).
Then, our \(Q(x) = a_0 + a_1 x + a_2 x^2 + a_3 x^3\), so
\begin{align}
    a_0 &\equiv 0 \pmod{17} \\
    a_0 + a_1 + a_2 + a_3 + e_0 &\equiv 1 \pmod{17} \\
    a_0 + 2 a_1 + 4 a_2 + 8 a_3 + 4 e_0 &\equiv 8 \pmod{17} \\
    a_0 + 3 a_1 + 9 a_2 + 27 a_3 + 6 e_0 &\equiv 18 \pmod{17} \\
    a_0 + 4 a_1 + 16 a_2 + 64 a_3 + 8 e_0 &\equiv 32 \pmod{17} \\
    \Rightarrow
    \begin{bmatrix}
        1 & 0 & 0 & 0 & 0 \\
        1 & 1 & 1 & 1 & 1 \\
        1 & 2 & 4 & 8 & 4 \\
        1 & 3 & 9 & 27 & 6 \\
        1 & 4 & 16 & 64 & 8
    \end{bmatrix}
    \begin{bmatrix}
        a_0 \\
        a_1 \\
        a_2 \\
        a_3 \\
        e_0
    \end{bmatrix}
    &\equiv
    \begin{bmatrix}
        0 \\
        1 \\
        8 \\
        18 \\
        32 
    \end{bmatrix} \pmod{17} \\
    \Rightarrow
    \begin{bmatrix}
        a_0 \\
        a_1 \\
        a_2 \\
        a_3 \\
        e_0
    \end{bmatrix}
    =
    \begin{bmatrix}
        0 \\
        0 \\
        -4
        1 \\
        4
    \end{bmatrix}
\end{align}
So our "solutions" are
\begin{align}
    Q(x) &= 2x^2 - 2x = 2x (x - 1) \\
    E(x) &= x - 1 \\
    P(x) &= \frac{Q(x)}{E(x)} = 2x \\
    (P(0), P(1), P(2)) &= (0, 2, 4) \neq (0, 1, 4)
\end{align}

\subsection{}

\begin{theorem}
    The Hamming distance between two Reed-Solomon codes \(RS[5, 3]\) is at least \(3\).
\end{theorem}
\begin{proof}
    Assume that the Hamming distance between two codes in \(RS[5, 3]\) is \(< 3\).
    Assign the vectors \(\bm{c}_a \neq \bm{c}_b\) to be associated with polynomials \(P_a(x), P_b(x)\).
    By construction, \(\deg\{P_a(x)\} = \deg\{P_b(x)\} \leqslant 4\).
    Since the \(H(\bm{c}_a, \bm{c}_b) < 3\), they must be the same at \(5\) other points since all \(RS[5, 3]\) codes are length \(5\).
    This means that \(P_a(x) = P_b(x)\) for at least \(5\) points, which means they are identical since they are of degree \(4\).
    This is a contradiction since we assumed that \(\bm{c}_a \neq \bm{c}_b\).
\end{proof}

\begin{theorem}
    The codeword \((1, 1, 3, 7, 13)\) and non-codeword \((1, 1, 4, 7, 13)\) have a smaller Hamming distance than any codeword on \(RS[5, 3]\).
\end{theorem}
\begin{proof}
    By inspection, the Hamming distance is \(1\).
    Since \((1, 1, 4, 7, 13)\) is not a codeword, the above theorem does not apply.
    Therefore, all other codewords on \(RS[5, 3]\) have a distance at least \(3 > 1\).
\end{proof}

\subsection{}

\begin{theorem}
    On the field \(RS[m, n]\),
    \begin{equation}
        H(C', E) = \min_{E' \in RS[m, n]} (H(C', E'))
    \end{equation}
    where \(C'\) is the received codeword and \(E\) is the encoded message, and \(H\) is the Hamming distance function.
\end{theorem}
\begin{proof}
    The minimum Hamming distance between any valid codewords is \(m - n + 1\),
    so
    \begin{equation}
        H(C', E) = m - n + 1
    \end{equation}
\end{proof}

\section{Linearity}

\begin{theorem}
    The element-wise sum of two Reed-Solomon codewords is also a Reed-Solomon codeword.
\end{theorem}
\begin{proof}
    Given two polynomials \(p(x), q(x)\) associated with codewords \(\bm{a} = (a_0, a_1, \ldots, a_{n - 1})\) and \(\bm{b} = (b_0, b_1, \ldots, b_{n - 1})\),
    \begin{align}
        p(x) &= a_0 + a_1 x + \cdots + a_{n - 1} x^{n - 1} \\
        q(x) &= b_0 + b_1 x + \cdots + b_{n - 1} x^{n - 1} \\
        p(x) + q(x) &= (a_0 + b_0) + (a_1 + b_1) x + \cdots + (a_{n - 1} + b_{n - 1}) x^{n - 1} \iff (a_0 + b_0, a_1 + b_1, \ldots, a_{n - 1} + b_{n - 1}) = \bm{a} + \bm{b}
    \end{align}
    This is true since the polynomials are closed under addition, so since the degree is preserved, the message length is also preserved, so this is yet another valid Reed-Solomon codeword.
\end{proof}

\stepcounter{section}

\section{Maze}

\subsection{}

Since there are \(8\) possible paths, the path must consist of exactly \(4\) \(x\)-axis moves and \(4\) \(y\)-axis moves, but since the total number of moves must be \(8\), we can classify this as a problem without replacement without order, so the total number is \(\binom{8}{4} = 70\).

\subsection{}

By construction of the problem, the first move must be an addition to \(x\), and the last move must be an addition to \(y\).
Therefore, we can reduce the search space to \(\binom{6}{3}\).
The list of possible rejected points is
\begin{align}
    (0, 1) \\
    (0, 2) \\
    (0, 3) \\
    (0, 4) \\
    (1, 2) \\
    (1, 3) \\
    (1, 4) \\
    (2, 3) \\
    (2, 4) \\
    (3, 4) \\
\end{align}
The total number of paths is thus \(14\).

\subsection{}

We can imagine a left parenthesis as an \(x\) move and right parenthesis as a \(y\) move, a mismatch corresponds to crossing the line \(y = x\).
Similarly, the first and last parentheses are pre-determined.
Thus, the answer is \(70 - 14 = 56\), since we eliminate all the legal moves from the previous question.

\subsection{}

\section{Good Game}

\begin{enumerate}
    \item Sending the maximum number of GGs means winning the fewest number of times.
    This means Maru must win the last games since they are worth the most points.
    The last three games are worth \(10 + 9 + 8 = 27\) points, so that is enough to tie.
    Maru has to win only one more game, so the max number of GGs is \(10 - 4 = 6\).
    \item Firstly, the cases \(\binom{10}{10}, \binom{10}{9}, \binom{10}{8}, \binom{10}{7}\) will always work since at worst case \(\sum_{i = 1}^7 i = 28 > 27\).
    Then, we can exclude \(\binom{10}{0}, \binom{10}{1}, \binom{10}{2}, \binom{10}{3}\) since at best \(10 + 9 + 8 = 27\).
    Considering \(\binom{10}{5}\), since the amount of games Maru wins or loses is even, if Maru loses some sequence, he wins the inverse of that sequence, so we need only divide by two.
    Then, considering either \(\binom{10}{6}, \binom{10}{4}\), we can cherry-pick the games Maru wins and combine them into a single sequence, and by the same symmetric argument as above, we can choose either one since \(\binom{10}{6} = \binom{10}{4}\).
    Therefore, the total combinations of winning games for Maru is \(\binom{10}{6} + \frac{1}{2} \binom{10}{5} + \sum_{i = 7}^{10} \binom{10}{i} = 512\).
\end{enumerate}

\section{Counting Functions}

\subsection{}

We can interpret this as a stars and bars with only one bar, so we can choose \(k\) items to put in the first bin, so \(n - k\) items must go into the second min.
Since the order within the bins doesn't matter,
\begin{equation}
    g(n) = \sum_{i = 1}^{n - 1} \binom{n}{i}
\end{equation}

\subsection{}

This problem is left as an exercise to the reader.

\subsection{}

This problem is left as an exercise to the reader.

\section{Sundry}

I did homework with:
\begin{itemize}
    \item Manny Fuentes \href{mailto:mfuentes@berkeley.edu}{mfuentes@berkeley.edu}
    \item Avery Perez \href{mailto:aj_perez@berkeley.edu}{aj\_perez@berkeley.edu}
    \item Victor Zhang \href{mailto:vzhang6@berkeley.edu}{vzhang6@berkeley.edu}
    \item Alex Chuang \href{mailto:alexc0413@berkeley.edu}{alexc0413@berkeley.edu}
\end{itemize}

\end{document}
