\documentclass{article}
\usepackage{amsmath, amssymb, amsfonts, amsthm}
\usepackage{cancel}
\usepackage[output-complex-root=j]{siunitx}
\usepackage[american, nooldvoltagedirection]{circuitikz}
\usepackage{bm}
\usepackage{listings}
\usepackage{graphicx}
\usepackage{fullpage}
\usepackage{hyperref}

\renewcommand{\thesection}{\arabic{section}}
\renewcommand{\thesubsection}{\thesection.\alph{subsection}}
\renewcommand{\thesubsubsection}{\thesubsection.\roman{subsubsection}}

\newtheorem{theorem}{Theorem}

\newcommand{\N}{\mathbb{N}}
\newcommand{\Z}{\mathbb{Z}}
\newcommand{\Q}{\mathbb{Q}}
\newcommand{\R}{\mathbb{R}}
\renewcommand{\C}{\mathbb{C}}
\newcommand{\unit}[1]{\bm{\hat{#1}}}
\newcommand{\iprod}[2]{\left\langle #1, #2 \right\rangle}
\newcommand{\tpose}[1]{\left[#1\right]^{\! \top} \!\!}
\newcommand{\diff}[1]{\frac{d}{d #1}}

\lstset{
    language=Python,
    tabsize=4,
    basicstyle=\ttfamily,
    numbers=left,
    numberstyle=\ttfamily,
    keywordstyle=\color{blue},
    frame=single
}

\title{CS70 HW02}
\author{Bryan Ngo}
\date{2020-09-07}

\begin{document}

\maketitle

\section{Induction}

\subsection{}

\begin{theorem}
    For all natural numbers \(n > 2\), \(2^n > 2n + 1\).
\end{theorem}
\begin{proof}
    Base case: \(n = 3\). \(2^3 > 2(3) + 1 \to 8 > 6\).
    Inductive hypothesis: \(2^k > 2k + 1\).
    \begin{align}
        2^{k + 1} &= 2 \cdot 2^k \\
        &> 2 (2k + 1) \\
        &= 4k + 2 \\
        &> 2k + 2 \\
        &= 2(k + 1) \\
    \end{align}
\end{proof}

\subsection{}

\begin{theorem}
    For all positive integers \(n\), \(1^2 + 2^2 + \cdots + n^2 = \frac{n (n + 1) (2n + 1)}{6}\).
\end{theorem}
\begin{proof}
    Base case: \(n = 1\). \(1^2 = \frac{1 \cdot 2 \cdot 3}{6} = 1\).
    Inductive hypothesis: \(\sum_{i = 1}^k i^2 = \frac{k (k + 1) (2k + 1)}{6}\).
    \begin{align}
        \sum_{i = 1}^{k + 1} i^2 &= \left(\sum_{i = 1}^k i^2\right) + (k + 1)^2 \\
        &= \frac{k (k + 1) (2k + 1)}{6} + (k + 1)^2 \\
        &= \frac{k (k + 1) (2k + 1) + 6(k + 1)^2}{6} \\
        &= \frac{(k + 1) (k (2k + 1) + 6(k + 1))}{6} \\
        &= \frac{(k + 1) (2k^2 + 7k + 6)}{6} \\
        &= \frac{(k + 1) (k + 2) (2k + 3)}{6}
    \end{align}
\end{proof}

\subsection{}

\begin{theorem}
    For all positive natural numbers \(n\), \(\frac{5}{4} \cdot 8^n + 3^{3n - 1}\) is divisible by \(19\).
\end{theorem}
\begin{proof}
    Base case: \(n = 1\). \(19 | (10 + 9)\).
    Inductive hypothesis: \(5 \cdot 2^{3k + 1} + 3^{3k - 1} = 19q\) for \(q \in \Z\).
    \begin{align}
        5 \cdot 2^{3 (k + 1) + 1} + 3^{3 (k + 1) - 1} &= 5 \cdot 2^{3k + 4} + 3^{3k + 2} \\
        &= 5 \cdot 8 \cdot 2^{3k + 1} + 27 \cdot 3^{3k - 1} \\
        &= 5 \cdot 8 \cdot 2^{3k + 1} + (8 + 19) \cdot 3^{3k - 1} \\
        &= 8 (5 \cdot 2^{3k + 1} + 3^{3k - 1}) + 19 \cdot 3^{3k - 1} \\
        &= 8 \cdot 19q + 19 \cdot 3^{3k - 1} \\
        &= 19 (8q + 3^{3k - 1})
    \end{align}
\end{proof}

\section{Negative Pacman Returns}

\begin{theorem}
    Pacman is able to reach \((0, 0)\) from any \((i, j) \in \N^2\) in finite time.
\end{theorem}

\begin{proof}
    Base case: Pacman is at \((0, 0)\), so no 
\end{proof}

\section{Sundry}

I worked on this homework by myself.

\end{document}
