\documentclass{article}
\usepackage{amsmath, amssymb, amsfonts, amsthm}
\usepackage{cancel}
\usepackage[output-complex-root=j]{siunitx}
\usepackage[american, nooldvoltagedirection]{circuitikz}
\usepackage{bm}
\usepackage{listings}
\usepackage{graphicx}
\usepackage{fullpage}
\usepackage{hyperref}

\renewcommand{\thesection}{\arabic{section}}
\renewcommand{\thesubsection}{\thesection.\alph{subsection}}
\renewcommand{\thesubsubsection}{\thesubsection.\roman{subsubsection}}
\newcommand{\lemmaautorefname}{Lemma}
\renewcommand{\labelenumi}{\alph{enumi}.}
\renewcommand{\labelenumii}{\roman{enumii}.}

\newtheorem{theorem}{Theorem}
\newtheorem{lemma}{Lemma}

\newcommand{\N}{\mathbb{N}}
\newcommand{\Z}{\mathbb{Z}}
\newcommand{\Q}{\mathbb{Q}}
\newcommand{\R}{\mathbb{R}}
\renewcommand{\C}{\mathbb{C}}
\newcommand{\unit}[1]{\bm{\hat{#1}}}
\newcommand{\iprod}[2]{\left\langle #1, #2 \right\rangle}
\newcommand{\tpose}[1]{\left[#1\right]^{\! \top} \!\!}
\newcommand{\diff}[1]{\frac{d}{d #1}}

\lstset{
    language=Python,
    tabsize=4,
    basicstyle=\ttfamily,
    numbers=left,
    numberstyle=\ttfamily,
    keywordstyle=\color{blue},
    frame=single
}

\title{CS 70 HW 09}
\author{Bryan Ngo}
\date{2020-10-30}

\begin{document}

\maketitle

\section{Unprogrammable Programs}

\subsection{}

\begin{theorem}
    A program \(P(F, x, y)\) that returns true if \(F(x) = y\) and false otherwise cannot exist.
\end{theorem}
\begin{proof}
    Consider the following Python program:
    \begin{lstlisting}
def one_a(f, x, y):
    if f(x) == y:
        return True
    elif f(x) != y or halts(f, x):
        return False
    \end{lstlisting}
    There are three cases to consider:
    \begin{itemize}
        \item If \(F(x) = y\), return true.
        \item If \(F(x) \neq y\), return false.
        \item If \(F(x)\) goes into an infinite loop, then we have to return false, since it never returns anything to begin with.
    \end{itemize}
    In order to resolve the last case, we require a function that tells us whether or not \(F(x)\) halts.
    This is a violation of the halting problem, so \(P\) cannot exist.
\end{proof}

\subsection{}

\begin{theorem}
    A program \(P(F, G)\) that returns true if \(F(x) = G(x)\) for some \(x \in X\) cannot exist.
\end{theorem}
\begin{proof}
    Knowing whether \(F\) and \(G\) halt on the same set of inputs requires knowing if it will halt on any given input in that set.
    In order to accomplish this, \(P\) must be able to solve the halting problem, since \(F\) and \(G\) are general functions.
    Therefore, by the halting problem, \(P\) cannot exist.
\end{proof}

\section{Computations on Programs}

\subsection{}

\begin{theorem}
    It is not possible to write a program \(P(n)\) that determines the shortest arithmetic formula that produces \(n\).
\end{theorem}
\begin{proof}
    Consider the following program:
    \begin{lstlisting}
def one_b(fs, n):
    min_f = lambda x: n
    min_len = func_len(min_f, n)
    for f in fs:
        if func_len(f) < min_len:
            min_f = f
            min_len = func_len(f)
    return min_f
    \end{lstlisting}
    where \lstinline|func_len| is a function that finds the length of any function in characters.
    This program works since we can place an upper bound on the length of our function: simply returning \(n\) itself, since it does not make sense to return a program that is longer than \lstinline|len(n)|.
    Therefore, this does not violate the halting problem since we can simply discard a function from the search once its length exceeds that of \lstinline|len(n)|.
\end{proof}

\subsection{}

\begin{theorem}
    There exists an algorithm to find the minimal program \(P(n)\) that prints out \(n \in \N\).
\end{theorem}
\begin{proof}
    Consider the following program:
    \begin{lstlisting}
def two_a(fs, n):
    min_f = lambda x: n
    min_t = perf(min_f, n)
    for f in fs:
        if f(n) == n and perf(f, n) < min_t:
            min_f = f
            min_t = perf(f, n)
    return min_f
    \end{lstlisting}
    where \lstinline|perf| measures the performance of the function.
    This program can correctly calculate the minimal program, considering two cases:
    \begin{itemize}
        \item The program being measured halts within \lstinline|min_t| steps.
        This means that it is better than the current minimal program, meaning it will replace \lstinline|min_f|.
        \item The program being measured does not halt within \lstinline|min_t| steps.
        This means halts beyond \lstinline|min_t| or does not halt at all, and will not replace \lstinline|min_f|.
    \end{itemize}
    This does not violate the halting problem because it is entirely possible to determine the halting of a program in a finite \(t\) steps.
    In our case, \(t\) is simply \lstinline|min_t|, which will always be finite since there will always exist a finite-halt program: simply returning \(n\) itself.
\end{proof}

\section{Kolmogorov Complexity}

\subsection{}

The notion of a smallest integer that cannot be defined in under \(280\) characters does not make sense because \(280\) is itself under that limit.

\subsection{}

\begin{theorem}
    Given the Kolmogorov complexity of a string \(n\) of length \(L\) compressed to \(x\) with length as \(K(x)\),
    \begin{equation}
        K(x) \geqslant L
    \end{equation}
\end{theorem}
\begin{proof}
    If \(K(n) < L\), then there will be multiple \(n\)'s that map to the same \(x\), by the pigeonhole principle.
    This means that it is impossible to decompress \(x\) back to its original string, violating the motivation behind lossless compression.
\end{proof}

\subsection{}

\begin{lstlisting}
def p(n):
    return max(i for i in range(2**n - 1), key=K)
\end{lstlisting}

\subsection{}

\begin{theorem}
    The program \(P\) that returns the maximum Kolmogorov complexity of a string \(n\) cannot exist.
\end{theorem}
\begin{proof}
    Note that \(K(P) = m\).
    Consider calculating \(P(n)\) for some \(n > m\).
    Then, \(P(n) \geqslant L(n)\) by \textbf{4.b}.
    This is a contradiction since by definition, the Kolmogorov complexity requires an optimally-compressed \(n\), but \(P(n)\) outputs a string that it should not be able to.
\end{proof}

\section{Five Up}

\begin{enumerate}
    \item \(|\Omega| = 2^5 = 32\)
    \item \(|H = 3| = \binom{5}{3} = 10\)
    \item \(|H \geqslant 3| = \sum_{i = 3}^5 \binom{5}{i} = 16\)
    \item \(P(\{HHHTT\}) = \frac{1}{32}\), \(P(\{HHHHT\}) = \frac{1}{32}\)
    \item Since there is only one possibility with no heads, \(P(H \geqslant 1) = \frac{31}{32}\)
    \item \(P(H \geqslant 3) = \frac{1}{32} \sum_{i = 3}^5 \binom{5}{i} = \frac{1}{2}\)
    \item \(P(\{HHHTT\}) = \left(\frac{2}{3}\right)^3 \left(\frac{1}{3}\right)^2 = \frac{8}{243}\), \(P(\{HHHHT\}) = \left(\frac{2}{3}\right)^4 \left(\frac{1}{3}\right) = \frac{16}{243}\)
    \item Since there is only one possibility with no heads, \(P(H \geqslant 1) = 1 - \left(\frac{1}{3}\right)^5 = \frac{242}{243}\)
    \item \(P(H \geqslant 3) = \sum_{i = 3}^5 \binom{5}{i} \left(\frac{2}{3}\right)^i \left(\frac{1}{3}\right)^{5 - i} = \frac{64}{81}\)
\end{enumerate}

\section{Ball-and-Bin Counting Problems}

\begin{enumerate}
    \item \(|\Omega| = \binom{7 + 5 - 1}{7} = 330\), \(|B_1 = 3| = \binom{4 + 4 - 1}{4} = 35\), \(P(B_1 = 3) = \frac{7}{66}\)
    \item \(P(B_1 \geqslant 3) = P(B_3 \geqslant 3) = \frac{1}{330} \sum_{i = 3}^7 \binom{(7 - i) + 4 - 1}{7 - i} = \frac{7}{33}\)
    \item \(P(B_n \geqslant 3) = \frac{1}{330} \left(5 \cdot 35 - 3(1 + 2 + 3 + 4)\right) = \frac{145}{330}\), where we subtract overcounting of bins that already have the requisite number of balls.
\end{enumerate}

\section{Monty Hall's Revenge}

\subsection{}

The original probability before revealing the door is \(P(C) = \frac{1}{n}\).
After revealing one of the goats, the original door still has a probability of \(P(C) = \frac{1}{n}\).
The number of possible cars is \(1 - \frac{1}{n} = \frac{n - 1}{n}\).
The total amount of doors remaining (that you have not chosen) is \(n - 2\).
Thus, the odds when switching are now \(\frac{n - 1}{n (n - 2)}\).

\subsection{}

In this situation, there will always be \(2\) remaining: the one that you chose, and the one you should switch to.
Then, our formula is now \(\frac{n - 1}{n (n - (n - 1))} = \frac{n - 1}{n}\).

\subsection{}

In order to maximize the net benefit of switching, you need as high a \(j\) as possible.
The chances of winning a car are \(\frac{k}{n}\).
The odds of choosing a door with a car is \(\frac{k (n - 1)}{n}\).
The odds after switching are \(\frac{k (n - 1)}{n (n - j - 1)}\).
Thus, the ratio of staying over switching are \(\frac{n - 1}{n - j - 1}\).
Thus, \(k\) is irrelevant, and you want the highest \(j\) possible, or \(j = n - k - 1\).

\end{document}
