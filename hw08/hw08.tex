\documentclass{article}
\usepackage{amsmath, amssymb, amsfonts, amsthm}
\usepackage{cancel}
\usepackage[output-complex-root=j]{siunitx}
\usepackage[american, nooldvoltagedirection]{circuitikz}
\usepackage{bm}
\usepackage{listings}
\usepackage{graphicx}
\usepackage{fullpage}
\usepackage{xcolor}
\usepackage{hyperref}

\renewcommand{\thesection}{\arabic{section}}
\renewcommand{\thesubsection}{\thesection.\alph{subsection}}
\renewcommand{\thesubsubsection}{\thesubsection.\roman{subsubsection}}
\newcommand{\lemmaautorefname}{Lemma}

\newtheorem{theorem}{Theorem}
\newtheorem{lemma}{Lemma}

\newcommand{\N}{\mathbb{N}}
\newcommand{\Z}{\mathbb{Z}}
\newcommand{\Q}{\mathbb{Q}}
\newcommand{\R}{\mathbb{R}}
\renewcommand{\C}{\mathbb{C}}
\newcommand{\unit}[1]{\bm{\hat{#1}}}
\newcommand{\iprod}[2]{\left\langle #1, #2 \right\rangle}
\newcommand{\tpose}[1]{\left[#1\right]^{\! \top} \!\!}
\newcommand{\diff}[1]{\frac{d}{d #1}}

\lstset{
    language=Python,
    tabsize=4,
    basicstyle=\ttfamily,
    numbers=left,
    numberstyle=\ttfamily,
    keywordstyle=\color{blue},
    frame=single
}

\title{CS 70 HW08}
\author{Bryan Ngo}
\date{2020-10-22}

\begin{document}

\maketitle

\section{Countability Proof Practice}

\subsection{}

\begin{theorem}
    The disjoint set of unit disks centered on \((x_0, y_0)\) with radius \(r > 0\) on \(\R^2\) is countable.
\end{theorem}
\begin{proof}
    \begin{lemma}
        \(|\Q \times \Q| = |\N \times \N| = \aleph_0\).
    \end{lemma}
    \begin{proof}
        By the pairing argument, it is known that \(|\N \times \N| = \aleph_0\).
        Then, by a similar separate pairing argument, \(|\Q| = \aleph_0\).
        Thus, the set of all \(2\)-tuples of rationals can be enumerated similarly to the \(2\)-tuples of naturals.
    \end{proof}
    Using the fact that a rational number exists between any two real numbers, we can uniquely identify any unit disk by a pair of rationals.
    Namely, we can use \((m, n)\) where \(m\) is the smallest rational between \((x_0, y_0)\) and \(n\) is the arithmetic mean of the rational between \((x_0, r)\) and \((y_0, r)\).
    Since for any unit disk, this is unique, we have established an injective function \(f: \{(x, y) \in \R^2 | (x - x_0)^2 + (y - y_0)^2 \leqslant r^2\} \mapsto (\Q \times \Q)\).
    By the above lemma, this means the set is countable.
\end{proof}

\subsection{}

\begin{theorem}
    The disjoint set of unit circles centered on \((x_0, y_0)\) with radius \(r > 0\) on \(\R^2\) is potentially uncountable.
\end{theorem}
\begin{proof}
    Consider the set of unit circles where every circle shares the same center \((x_0, y_0)\).
    Then, since the set is disjoint by definition, for any two set elements with radii \(r_{i, j}\), \(r_i \neq r_j\).
    Since \(r \in \R\), our set of unit circles is a potentially infinite subset of \(\R\).
    By diagonalization, this set is potentially uncountable.
\end{proof}

\subsection{}

\begin{theorem}
    The set of increasing functions \(f: \N \mapsto \N\) such that if \(x \geqslant y\), \(f(x) \geqslant f(y)\) is potentially uncountable.
\end{theorem}
\begin{proof}
    Consider the following diagonalization argument for the set of all increasing functions on \(\N\):
    \begin{equation}
        \begin{array}[]{c|cccc}
            & 0 & 1 & 2 & \cdots \\
            \hline
            f_1 & {\color{red} 0} & 1 & 2 & \cdots \\
            f_2 & 2 & {\color{red} 4} & 8 & \cdots \\
            f_3 & 2 & 3 & {\color{red} 5} & \cdots \\
            \vdots & \vdots & \vdots & \vdots & \ddots \\
            f_n & 1 & 5 & 6 & \cdots
        \end{array}
    \end{equation}
    where \(f_n(x) = x + 1\); note that \(f_n\) itself is increasing.
    Since \(f_n\) differs from every increasing function, it cannot be an element of the set of all increasing functions on \(\N\).
    By contradiction, the set is uncountable.
\end{proof}

\subsection{}

\begin{theorem}
    The set of decreasing functions \(f: \N \mapsto \N\) such that if \(x \geqslant y\), \(f(x) \leqslant f(y)\) is countable.
\end{theorem}
\begin{proof}
    For any function \(f \in F\), where \(F\) is the set in question, the number of potential values is finite.
    This is because we must start at some natural number \(p\), which is finite, and end at some \(q \in [0, p]\), since \(0\) is the smallest natural.
    Consider the function \(g: (\N \mapsto \N) \mapsto \N \mapsto \{0, 1\}\), which takes in a function \(f\) and natural \(x\), then outputs \(1\) if it decreased from the previous natural, and \(0\) otherwise; \(g(f, 0) = 0\) as a special case.
    Then, performing the mapping on our set yields a set of finite binary strings, which is always countable.
\end{proof}

\section{Hilbert's Hotel}

\subsection{}

For any \(k\) guests, we simply move each existing guest at room \(n\) to room \(n + k\), then fill the new guests into the vacancies.

\subsection{}

For a countably infinite new amount of guests, we simply move each guest at room \(n\) to room \(2n\), then move each new guest to room \(2n + 1\).

\subsection{}

For a countably infinite amount of countably infinite guests, we move each current guest in room \(n\) to room \(2^n\).
Then, for the \(a\)-th bus, we move the \(b\)-th guest to room \(p_a^b\), where \(p_a\) is the \(a\)-th prime number greater than \(2\).

\end{document}
